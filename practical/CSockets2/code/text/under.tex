\chapter{Under the Hood}
\label{chap:under}%
\newcommand{\netstat}{\texttt{netstat}}

Some of the subtleties of network programming are difficult to grasp
without some understanding of the data structures associated with each
socket in the implementation and certain details of how the underlying
protocols work.  This is especially true of stream (TCP) sockets.
This chapter describes some of what goes on ``under the hood'' when you
create and use a socket.
The initial discussion and Section~\ref{sect:demux} apply to both
datagram (UDP) and stream (TCP) sockets; the rest applies only to TCP
sockets.
Please note that
this description covers only the normal sequence of events and glosses
over many details.  Nevertheless, we believe that even this basic
level of understanding is helpful.  Readers who want the full story
are referred to the TCP specification~[\cite{RFC793}] or to one of the
more comprehensive treatises on the subject~[\cite{ComerV2},
\cite{StevensV2}].

Figure~\ref{fig:structures} is a simplified view of some of the
information associated with a socket---that is, the object created by
a call to \fcn{socket()}.  The integer returned by \fcn{socket()} is
best thought of as a ``handle'' that identifies the collection of data
structures for one communication endpoint that we refer to in this
chapter as the ``socket structure.''
As the figure indicates, more than one
descriptor can refer to the same socket structure.  In fact,
descriptors in \emph{different processes} can refer to the same
underlying socket structure.

\begin{figure}
%\jfigs{figures/ja05f01}{0.7}
\jfigs{figures/structures.eps}{0.5\textwidth}
\caption{\label{fig:structures}Data structures associated with a socket.}
\end{figure}

By ``socket structure'' here we mean all data structures
in the socket layer and TCP implementation that contain state
information relevant to this socket abstraction. 
Thus, the socket structure contains send and receive
queues and other information, including the following:


\begin{itemize}

\item The {local and remote} Internet addresses and port numbers
associated with the socket.  The local Internet address (labeled
``Local IP'' in the figure) is one of those assigned to the local
host; the local port is set at \fcn{bind()} time.
The remote address and port identify the remote
socket, if any, to which the local socket is connected.  We will say
more about how and when these values are determined shortly
(Section~\ref{sect:demux} contains a concise summary).

\item A FIFO queue (``\sque'') of received data waiting to be delivered and a
FIFO queue (``\rque'') for data waiting to be transmitted.

\item For a TCP socket, additional {protocol state} information
relevant to the opening and closing TCP handshakes.  In
Figure~\ref{fig:structures}, the state is ``Closed''; all sockets start
out in the Closed state.
\end{itemize}
Some general-purpose operating systems provide tools that enable users
to obtain a ``snapshot'' of these underlying data structures.
On such tool is \netstat, which is typically available on both
Unix (Linux) and Windows platforms.  Given appropriate options, 
\netstat\ displays exactly the information indicated in
Figure~\ref{fig:structures}: number of bytes in 
\emph{SendQ} and \emph{RecvQ}, 
local and remote IP addresses and port
numbers, and the connection state.
Command-line options may vary, but the output should look something
like this:
\begin{verbatim}
Active Internet connections (servers and established)
Proto Recv-Q Send-Q Local Address           Foreign Address       State      
tcp        0      0 0.0.0.0:36045           0.0.0.0:*             LISTEN     
tcp        0      0 0.0.0.0:111             0.0.0.0:*             LISTEN     
tcp        0      0 0.0.0.0:53363           0.0.0.0:*             LISTEN     
tcp        0      0 127.0.0.1:25            0.0.0.0:*             LISTEN     
tcp        0      0 128.133.190.219:34077   4.71.104.187:80       TIME_WAIT  
tcp        0      0 128.133.190.219:43346   79.62.132.8:22        ESTABLISHED
tcp        0      0 128.133.190.219:875     128.133.190.43:2049   ESTABLISHED
tcp6       0      0 :::22                   :::*                  LISTEN
\end{verbatim}
The first four lines and the last line
depict server sockets listening for connections.
(The last line is a listening socket bound to an IPv6 address.)
The fifth line corresponds to a connection to a web server
(port 80) that is partially shut
down (see Section~\ref{sect:closingTCP} below).
The next-to-last two lines are existing TCP connections.
You may want to play with \netstat, if it is available on your system,
to examine the status of connections in the scenarios depicted below.
Be aware, however, that because the transitions between states
depicted in the figures happen so quickly, it may be difficult to
catch them in the ``snapshot'' provided by \netstat.

Knowing that these data structures exist and how they are affected by
the underlying protocols is useful because they control various
aspects of the behavior of the socket.
For example, because TCP
provides a \emph{reliable} byte-stream service, a copy of any data
sent over a TCP socket must be kept \emph{by the TCP implementation}
until it has been successfully received at the other end of the
connection.
Completion of a call to \fcn{send()} on a TCP socket
does \emph{not}, in general,
imply that the data has actually transmitted---only that it has
been copied into the local buffer.  Under normal conditions it will be
transmitted soon, but the exact moment is under the control of TCP,
not the application.
Moreover, the nature of the
byte-stream service means that message boundaries are \emph{not\/} necessarily
preserved in the input stream.  As we saw in
Section~\ref{sect:framing}, this means that most application protocols
need a \emph{framing\/} mechanism, so the receiver can tell when it has
received an entire message.

On the other hand, with a datagram (UDP) socket, 
packets are
\emph{not} buffered for retransmission, and by the time a call to
\fcn{send}/\fcn{sendto} returns, the data has been
given to the network subsystem for transmission. If the network
subsystem cannot handle the message for some reason, the packet is
silently dropped (but this is rare).

The next three sections deal with some of the subtleties of sending
and receiving with TCP's byte-stream service.  Then,
Section~\ref{sect:lifecycle} considers the connection establishment
and termination of the TCP protocol.  Finally,
Section~\ref{sect:demux} discusses the process of matching incoming
packets to sockets and the rules about binding to port numbers.

\section{Buffering and TCP}
\label{sect:sendrec}%

As a programmer, the most important thing to remember when using a TCP
socket is this:

\begin{quote}
\textbf{You cannot assume any correspondence between
writes to the output stream at one end of the connection
and reads from the input stream at  the other end.}
\end{quote}

In particular, data passed in a single invocation of the output
stream's \fcn{send} method at the sender can
be spread across multiple invocations of the input stream's
\fcn{recv} method at the other end; and a
single \fcn{recv} may return data passed in
multiple \fcn{send}s.

To see this,
consider a program that does the following:
%
\begin{inlinecode}
rv = connect(s,...);
...
rv = send(s, buffer0, 1000, 0);
...
rv = send(s, buffer1, 2000, 0);
...
rv = send(s, buffer2, 5000, 0);
...
close(s);

\end{inlinecode}
%
where the ellipses represent code that sets up the data in the
buffers but contains no other calls to \fcn{send}.
%
This TCP connection transfers 8000 bytes to the receiver.  The way
these 8000 bytes are grouped for delivery at the receiving end of the
connection depends on the timing between the calls to
\fcn{send()} and
\fcn{recv()} at the two ends of the
connection---as well as the size of the buffers provided to the
\fcn{recv()} calls.

We can think of the sequence of all bytes sent (in one direction) on a
TCP connection up to a particular instant in time as being divided into three
FIFO queues:

\begin{enumerate}

\item \emph{\sque}: Bytes buffered in the underlying implementation at
the sender that have been written to the output stream but not yet
successfully transmitted to the receiving host.

\item \emph{\rque}: Bytes buffered in the underlying implementation at
the receiver waiting to be delivered to the receiving program---that
is, read from the input stream.

\item \emph{Delivered}: Bytes already read from the input stream by
the receiver.

\end{enumerate}
%
A call to \fcn{send()} at the sender
appends bytes to \sque.  The TCP protocol is responsible for
moving bytes---in order---from \sque\ to \rque.  It is
important to realize that this transfer cannot be controlled or
directly observed by the user program, and that it occurs in chunks
whose sizes are more or less independent of the size of the buffers
passed in \fcn{send}s.  Bytes are moved from
\rque\ to \emph{Delivered}\ by calls to \fcn{recv()};
the size of the transferred chunks depends on the amount of
data in \rque\ and the size of the buffer given to
\fcn{recv()}.

Figure~\ref{fig:buffers0} shows one possible state of the three queues
\emph{after} the three \fcn{send}s in
the example above, but \emph{before} any
\fcn{in.reads}s at the other end.  The
different shading patterns denote bytes passed in the three different
invocations of \fcn{send()} shown above.

\begin{figure}
%\jfigs{figures/ja05f02}{0.7}
\jfigs{figures/buffers0.eps}{\textwidth}
\caption{\label{fig:buffers0}State of  the three queues after three
  \fcn{send()} calls.}
\end{figure}

The output of \netstat\ on the sending host at the instant depicted in
this figure would contain a line like:
\begin{verbatim}
Active Internet connections
Proto Recv-Q Send-Q Local Address        Foreign Address      State
tcp        0   6500 10.21.44.33:43346    192.0.2.8:22         ESTABLISHED
\end{verbatim}
On the receiving host, \netstat\ shows:
\begin{verbatim}
Active Internet connections
Proto Recv-Q Send-Q Local Address        Foreign Address      State
tcp     1500      0 192.0.2.8:22         10.21.44.33:43346    ESTABLISHED
\end{verbatim}

Now suppose the receiver calls \fcn{recv} with
a byte array of size 2000.  The \fcn{recv} call
will move all of the 1500 bytes present in the waiting-for-delivery
(\rque) queue into the byte array and return the value 1500.
Note that this data includes bytes passed in both the first and second
calls to \fcn{send}.  At some time later,
after TCP has completed transfer of more data, the three partitions
might be in the state shown in Figure~\ref{fig:buffers1}.

\begin{figure}
%\jfigs{figures/ja05f03}{0.7}
\jfigs{figures/buffers1.eps}{\textwidth}
\caption{\label{fig:buffers1}After first \fcn{recv}}
\end{figure}

If the receiver now calls \fcn{recv} with a
buffer of size 4000, that many bytes will be moved from the
waiting-for-delivery (\rque) queue to the already-delivered
(\emph{Delivered}) queue; this includes the remaining 1500 bytes from
the second \fcn{send}, plus the first 2500
bytes from the third \fcn{send}.  The
resulting state of the queues is shown in Figure~\ref{fig:buffers2}.

\begin{figure}
%\jfigs{figures/ja05f04}{0.7}
\jfigs{figures/buffers2.eps}{\textwidth}
\caption{\label{fig:buffers2}After another \fcn{recv}}.
\end{figure}

The number of bytes returned by the next call to
\fcn{recv} depends on the size of the buffer
and the timing of the transfer of data over the network from the
send-side socket/TCP implementation to the receive-side
implementation.  The movement of data from the \sque\ to the
\rque\ buffer has important implications for the design of
application protocols. We have already encountered the need to parse
messages as they are received via a socket
when in-band delimiters are used for framing (see
Section~\ref{sect:framing}).  In the following 
sections, we consider two more subtle ramifications.

\section{Deadlock Danger}
\label{sect:deadlock}%

Application protocols have to be designed with some care to avoid
\emph{deadlock}---that is, a state in which each peer is blocked
waiting for the other to do something.  For example, it is pretty
obvious that if both client and server try to receive immediately
after a connection is established, deadlock will result.  Deadlock can
also occur in less immediate ways.

The buffers \sque\ and \rque\ in the implementation have
limits on their capacity.  Although the actual amount of memory they
use may grow and shrink dynamically, a hard limit is necessary to
prevent all of the system's memory from being gobbled up by a single
TCP connection under control of a misbehaving program.  Because these
buffers are finite, they can fill up, and it is this fact, coupled
with TCP's \emph{flow control} mechanism, that leads to the
possibility of another form of deadlock.

Once \rque\ is full, the TCP {flow control} mechanism kicks in
and prevents the transfer of any bytes from the sending host's
\sque\ until space becomes available in \rque\ as a
result of the receiver calling \fcn{recv()}.
(The purpose of the flow
control mechanism is to ensure that the sender does not transmit more
data than the receiving system can handle.)  A sending program can
continue to call \fcn{send()} until \sque\ is full; however, once
\sque\ is full, a \fcn{send} will block until space
becomes available, that is, until some bytes are transferred to the
receiving socket's \rque.  If \rque\ is also full,
everything stops until the receiving program calls
\fcn{recv} and some bytes are transferred
to \emph{Delivered}.

Let's assume the sizes of \sque\ and \rque\ are
\sqsize\ and \rqsize, respectively.  A
call to \fcn{send()} passing in a buffer of size $n$
such that $n > \sqsize$ will not return until at least
$n-\sqsize$ bytes have been transferred to \rque\ at the
receiving host.  If $n$ exceeds $(\sqsize +\rqsize)$,
\fcn{send()} cannot return until after the
receiving program has read at least $n-(\sqsize +\rqsize)$ bytes
from the input stream.  If the receiving program does not call
\fcn{recv()}, a large
\fcn{send} may not complete successfully.  In
particular, if both ends of the connection call
\fcn{send()} simultaneously, each passing a buffer
bigger than $\sqsize + \rqsize$ bytes,
deadlock \emph{will\/} result: neither write will ever complete, and both
programs will remain blocked forever.

As a concrete example, consider a connection between a program on Host
A and a program on Host B.  Assume \sqsize\ and \rqsize\ are 500
at both A and B.  Figure~\ref{fig:deadlock} shows what happens when
both programs try to send 1500 bytes at the same time.  The first 500
bytes of data in the program at Host A have been transferred to the
other end; another 500 bytes have been copied into \sque\ at
Host A.  The remaining 500 bytes cannot be sent---and therefore
\fcn{send()} will not return---until
space frees up in \rque\ at Host B.  Unfortunately, the same
situation holds in the program at Host B.  Therefore, neither
program's call to \fcn{send()} call will ever return!

\begin{figure}
%\jfigs{figures/ja05f05}{0.7}
\jfigs{figures/deadlock3.eps}{\textwidth}
\caption{\label{fig:deadlock}Deadlock due to simultaneous \fcn{send}s
to output streams at opposite ends of the connection.}
\end{figure}

\begin{quote}
The moral of the story: Design the protocol carefully to avoid
sending large quantities of data simultaneously in both directions.
\end{quote}

% XXXX Removed because this example is only in the Java version! :-(

%% Can this really happen?  Let's review the compression protocol example
%% in Section~\ref{sect:shutdown}.  Try running the compression client
%% with a large file that is still large \emph{after compression}.  The
%% precise definition of ``large'' here depends on your system, but a
%% file that is already compressed and exceeds 2MB should do nicely.  For
%% each read/write, the compression client prints an ``R''/``W'' to the
%% console.  If both the uncompressed and compressed versions of the file
%% are large enough, your client will print a series of Ws and then stop
%% without terminating or printing any~Rs.

%% Why does this happen?  The program \file{CompressClient.java} sends
%% \emph{all} of the uncompressed data to the compression server
%% \emph{before} it attempts to read anything from the compressed stream.
%% The server, on the other hand, simply reads the uncompressed byte
%% sequence and writes the compressed sequence back to the client. (The
%% number of bytes the server reads before it writes some compressed data
%% depends on the compression algorithm it uses.)  Consider the case
%% where \sque\ and \rque\ for both client and server hold
%% 500 bytes each and the client sends a 10,000-byte (uncompressed) file.
%% Suppose also that for this file the server reads about 1000 bytes and
%% then writes 500 bytes, for a 2:1 compression ratio. After the client
%% sends 2000 bytes, the server will eventually have read them all and
%% sent back 1000 bytes, and the client's \rque\ and the server's
%% \sque\ will both be full.  After the client sends another 1000
%% bytes and the server reads them, the server's subsequent attempt to
%% write will block.  When the client sends the next 1000 bytes, the
%% client's \sque\ and the server's \rque\ will both fill up.
%% The next client write will block, creating deadlock.

%% How do we solve this problem?  One solution is to execute the
%% client writing and reading loop in separate threads.  One thread
%% repeatedly reads uncompressed bytes from a file and sends
%% them to the server until the end of the file is reached, whereupon it
%% calls \fcn{shutdown()} on the
%% socket.  The other thread repeatedly reads compressed
%% bytes from the input stream connected to the server
%% and writes them to the output file, until the
%% input stream ends (i.e., the server closes the socket).  When one
%% thread blocks, the other thread can proceed independently.  We can
%% easily modify our client to follow this approach by putting the call
%% to \fcn{SendBytes} in \file{CompressClient.java} inside a thread as
%% follows:

%% \begin{inlinecode}
%% Thread thread = new Thread() \{
%%   public void run() \{
%%     try \{
%%       SendBytes(sock, fileIn);
%%     \} catch (Exception ignored) \{\}
%%   \}
%% \};
%% thread.start();
%% \end{inlinecode}

%% \noindent See \file{CompressClientNoDeadlock.java} on the book's Web
%% site for the complete example.

%% Of course, the problem can be solved without using threads, through
%% the use of nonblocking \class{Channel}s and \class{Selector}s,
%% as described in Chapter~\ref{chap:nio}.

\section{Performance Implications}

The TCP implementation's need to copy user data into \sque\ before
sending it also has implications for performance.  In
particular, the sizes of the \sque\ and \rque\ buffers
affect the throughput achievable over a TCP connection.  ``Throughput''
is the \emph{rate} at which bytes of user data from the sender are
made available to the receiving program; in programs that transfer a
large amount of data, we want to maximize the number of bytes
delivered per second.  In the absence
of network capacity or other limitations, \emph{bigger buffers generally
result in higher throughput}.

The reason for this has to do with the cost of transferring data into
and out of the buffers in the underlying implementation.  If you want
to transfer $n$ bytes of data (where $n$ is large), it is generally
much more efficient to call \fcn{send()} once
with a buffer of size $n$ than it is to call it $n$ times with a
single byte.\footnote{The same thing generally applies to receiving,
although calling \fcn{recv} with a larger buffer does not
guarantee that more data will be returned---in general, only the
data present at the time of a call will be returned.}  However, if you call
\fcn{send()} with a size parameter that is much
larger than \sqsize\ (the size of \sque),
the system has to transfer the data from the
user address space in \sqsize-sized chunks.  That is, the socket
implementation fills up the \sque\ buffer, waits for data to be
transferred out of it by the TCP protocol, refills \sque, waits
some more, and so on.  Each time the socket implementation has to wait
for data to be removed from \sque, some time is wasted in the
form of overhead (a context switch occurs). This overhead is
comparable to that incurred by a completely new call to
\fcn{send()}.  Thus the \emph{effective\/} size
of a call to \fcn{send} is limited by the actual \sqsize.
The same thing applies at the receiving end:
however large the buffer we pass to
\fcn{recv()}, data will be copied out in chunks no
larger than \rqsize, with overhead incurred between chunks.

If you are writing a program for which throughput is an important
performance metric, you will want to change the send and receive
buffer sizes using the  \const{SO\_RCVBUF} and \const{SO\_SNDBUF} socket
options.   Although there is
always a system-imposed maximum size for each buffer, it is typically
significantly larger than the default on modern systems.  Remember
that these considerations apply only if your program needs to send an
amount of data significantly larger than the buffer size, all at once.
Note also that wrapping a TCP socket in a \type{FILE}-stream adds
another stage of buffering and additional overhead, and thus may
negatively affect throughput.

\section{TCP Socket Life Cycle}
\label{sect:lifecycle}%

When a new TCP \type{socket} is created, it cannot be used immediately
for sending and receiving data.  First it needs to be connected to a
remote endpoint.
%
Let us therefore consider in more detail how the underlying structure
gets to and from the connected, or ``Established,'' state.  As we'll
see later, these details affect the definition of
reliability and the ability to bind a socket to a particular port that
was in use earlier.

\subsection{Connecting}
\label{sect:connecting}%

The relationship between the \fcn{connect()} call and
the protocol events associated with
connection establishment at the client are
illustrated in Figure~\ref{fsm0}.  In this and the remaining figures
of this section, the large arrows depict external events that cause
the underlying socket structures to change state.  Events that occur
in the application program---that is, method calls and returns---are
shown in the upper part of the figure; events such as message arrivals
are shown in the lower part of the figure.  Time proceeds left to
right in these figures.  The client's Internet address is depicted as
A.B.C.D, while the server's is W.X.Y.Z; the server's port number is~Q.
(We have depicted IPv4 addresses, but everything here applies to both
IPv4 and IPv6.)
\begin{figure}
%\jfigs{figures/ja05f06}{0.7}
\jfigs{figures/fsm0.eps}{\textwidth}
\caption{\label{fsm0}Client-side connection establishment.}
\end{figure}

When the client calls \fcn{connect()}
with the server's Internet address, W.X.Y.Z, and port, Q,
the underlying implementation creates a socket instance; it is
initially in the Closed state.  If the client did not specify the
local address/port in the constructor call, a local port number (P),
not already in use by another TCP socket, is chosen by the
implementation.  The local Internet address is also assigned; if not
explicitly specified, the address of the network interface through
which packets will be sent to the server is used.  The implementation
copies the local and remote addresses and ports into the underlying
socket structure, and initiates the TCP connection establishment
handshake.

The TCP opening handshake is known as a \emph{3-way handshake} because
it typically involves three messages: a connection request from client
to server, an acknowledgment from server to client, and another
acknowledgment from client back to server.  The client TCP considers
the connection to be established as soon as it receives the
acknowledgment from the server.  In the normal case, this happens
quickly.  However, the Internet is a best-effort network, and either
the client's initial message or the server's response can get lost.
For this reason, the TCP implementation retransmits handshake messages
multiple times, at increasing intervals.  If the client TCP does not
receive a response from the server after some time, it \emph{times
out} and gives up.  In this case \fcn{connect()} returns $-1$ and sets
\var{errno} to \const{ETIMEDOUT}.
The implementation tries very hard to complete the connection before
giving up, and thus it can take on the
order of minutes for a \fcn{connect()} call to fail.
%
After the initial handshake message is sent and before the
reply from the server is received (i.e. the middle part of
Figure~\ref{fsm0}), the output from \netstat\ on the
client host would look something like:

\begin{verbatim}
Active Internet connections
Proto Recv-Q Send-Q Local Address    Foreign Address      State      
tcp        0      0 A.B.C.D:P        W.X.Y.Z:Q            SYN_SENT
\end{verbatim}
where ``\verb+SYN_SENT+'' is the technical name of the client's state
between the first and second messages of the handshake.

If the
server is not accepting connections---say, if there is no program
associated with the given port at the destination---the server-side
TCP will respond (immediately) with a rejection message instead of an
acknowledgment, and \fcn{connect()} returns $-1$ with \var{errno} set
to \const{ECONNREFUSED}.
%
Otherwise, after the client receives a positive reply from the server,
the netstat output would look like:
\begin{verbatim}
Active Internet connections
Proto Recv-Q Send-Q Local Address    Foreign Address      State      
tcp        0      0 A.B.C.D:P        W.X.Y.Z:Q            ESTABLISHED
\end{verbatim}

The sequence of events at the server side is rather different; we
describe it in Figures~\ref{fsm1a}, \ref{fsm1b}, and~\ref{fsm1c}.
The server needs to bind to the particular TCP port
known to the client. Typically, the server specifies only the
port number (here, Q)
in the \fcnsys{bind()} call and gives the special
wildcard address
\const{INADDR\_ANY} for the local IP address.
In case the server host has more than one IP address, this technique allows the
socket to receive connections addressed to any of its IP addresses.
When the server calls \fcn{listen()}, the state of the socket is changed to
``Listening'', indicating that it is ready to accept new connections.
%
These events are depicted in Figure~\ref{fsm1a}.
The output from \netstat\ on the server after this sequence
would include a line like:
\begin{verbatim}
Active Internet connections
Proto Recv-Q Send-Q Local Address    Foreign Address      State      
tcp        0      0 0.0.0.0:Q        0.0.0.0:0            LISTENING
\end{verbatim}
%
Note that any client
connection request that arrives at the server before the
call to \fcn{listen()} will be rejected, even if it arrives
after the call to \fcn{bind()}.

\begin{figure}
%\jfigs{figures/ja05f07}{0.7}
\jfigs{figures/fsm1a.eps}{\textwidth}
\caption{\label{fsm1a}Server-side socket setup.}
\end{figure}

\begin{figure}
%\jfigs{figures/ja05f08}{0.7}
\jfigs{figures/fsm1b.eps}{\textwidth}
\caption{\label{fsm1b}Incoming connection request processing.}
\end{figure}


The next thing the server does is call \fcnsys{accept()}, which
blocks until a connection with a client is established.
We therefore focus in Figure~\ref{fsm1b} on the events that
occur in the TCP implementation when a client connection request
arrives.  Note that everything depicted in this figure happens
``under the covers,'' in the TCP implementation.

When the request for a connection arrives from the client,
a new socket structure is created for the connection.
The new socket's addresses are filled
in based on the arriving packet: The packet's destination  Internet
address and port (W.X.Y.Z and Q, respectively)  become the socket's local
address and port; the packet's source
address and port (A.B.C.D and P)
become the  socket's remote Internet address and port.
Note that the local port number  of the new socket
is always the same as that of the listening socket.
The new socket's state is set to Connecting, and it is added to a list of
not-quite-connected sockets associated with the original server socket.
Note well that the original server socket does not change state.
%
At this point the output of \netstat\ should show \emph{both\/} the original,
listening socket and the newly-created one:
\begin{verbatim}
Active Internet connections
Proto Recv-Q Send-Q Local Address    Foreign Address      State      
tcp        0      0 0.0.0.0:Q        0.0.0.0:0            LISTENING
tcp        0      0 W.X.Y.Z:Q        A.B.C.D:P            SYN_RCVD
\end{verbatim}

In addition to creating a new underlying socket structure, the
server-side TCP implementation sends an acknowledging TCP handshake
message back to the client.  However, the server TCP does not consider
the handshake complete until the third message of the 3-way handshake
is received from the client.  When that message eventually arrives,
the new structure's state is set to ``Established'', and it is then
(and only then) moved to a list of socket structures associated with
the original socket, which represent established connections ready to be
\fcn{accept}ed.
(If the third
handshake message fails to arrive, eventually the ``Connecting''
structure is deleted.)
Output from \netstat\ would then include:
\begin{verbatim}
Active Internet connections
Proto Recv-Q Send-Q Local Address    Foreign Address      State      
tcp        0      0 0.0.0.0:Q        0.0.0.0:0            LISTENING
tcp        0      0 W.X.Y.Z:Q        A.B.C.D:P            ESTABLISHED
\end{verbatim}

Now we can consider (in Figure~\ref{fsm1c}) what happens when the
server program calls \fcn{accept()}.
The call unblocks as soon as there is something in the listening
socket's list of new connections. 
(Note that this list may already be non-empty when \fcn{accept()}
is called.)  At that time, the new socket structure is removed from the list,
and a socket descriptor is allocated and returned as the result of
\fcn{accept()}.

\begin{figure}
%\jfigs{figures/ja05f09}{0.7}
\jfigs{figures/fsm1c.eps}{\textwidth}
\caption{\label{fsm1c}\fcn{accept}}
\end{figure}

It is important to note that each structure in the
server socket's associated list
represents a fully established TCP connection with a client at the
other end.  Indeed, the client can send data as soon as it receives
the second message of the opening handshake---which may be long before
the server accepts the client connection!

\subsection{Closing a TCP Connection}
\label{sect:closingTCP}%

TCP has a \emph{graceful close} mechanism that allows applications to
terminate a connection without having to worry about loss of data that
might still be in transit.  The mechanism is also designed to allow
data transfers in each direction to be terminated independently.
% XXXXX FIXME: COMPRESSION EXAMPLE
% as in
% the [FIX: compression] example of Section~\ref{sect:shutdown}.
It works like
this: the application indicates that it is finished sending data on a
connected socket by calling \fcn{close()} or by
calling \fcn{shutdown()}.  At
that point, the underlying TCP implementation first transmits any data
remaining in \sque\ (subject to available space in
\rque\ at the other end), and then sends a closing TCP
handshake message to the other end.  This closing handshake message
can be thought of as an end-of-stream marker: it tells the
receiving TCP that no more bytes will be placed in \rque.
(Note that the closing handshake message itself is \emph{not} passed
to the receiving application, but that its position in the byte stream
is indicated by \fcn{recv} returning $-1$.)  The closing TCP waits
for an acknowledgment of its closing handshake message, which
indicates that all data sent on the connection made it safely to
\rque.  Once that acknowledgment is received, the connection is
``Half closed.''
%
The connection is not \emph{completely} closed until a symmetric
handshake happens in the other direction---that is, until \emph{both}
ends have indicated that they have no more data to send.

\begin{figure}
%\jfigs{figures/ja05f10}{0.7}
\jfigs{figures/fsm2.eps}{\textwidth}
\caption{\label{fsm2}Closing a TCP connection first.}
\end{figure}

The closing event sequence in TCP can happen in two ways: either one
application calls \fcn{close()} (or
\fcn{shutdown()}) and completes
its closing handshake before the other calls
\fcn{close/shutdown}, or both close
simultaneously, so that their
closing handshake messages cross in the network.  Figure~\ref{fsm2}
shows the sequence of events in the implementation when the
application invokes \fcn{close()} \emph{before\/}
the other end closes.  The closing handshake message is sent, the
state of the socket structure is set to ``Closing'', and the call
returns.
After this point, further attempts to perform any operation on the
socket result in error returns. % XXXX
When the acknowledgment for the close handshake is
received, the state changes to ``Half closed'', where it remains until
the other end's close handshake message is received.
At this point the output of \netstat\ on the
client would show the status of the connection as:
\begin{verbatim}
Active Internet connections
Proto Recv-Q Send-Q Local Address    Foreign Address      State      
tcp        0      0 A.B.C.D:P        W.X.Y.Z:Q            FIN_WAIT_2
\end{verbatim}
(\verb+FIN_WAIT_2+ is the technical name for the ``half-closed'' state at
the host that initiates close first.  The state denoted by ``closing''
in the figure is technically called \verb+FIN_WAIT_1+,
but it is transient and is difficult to catch with \netstat.)
Note that if the
remote endpoint goes away while the connection is in this state, the
local underlying structure will stay around indefinitely.  Otherwise,
when the other end's close handshake message arrives an
acknowledgment is sent and the state is changed to ``Time-Wait''.
Although the descriptor in the application
program may have long since been reclaimed (and even reused),
the associated underlying
socket structure continues to exist in the implementation for a minute or
more; the reasons for this are discussed on
page~\pageref{time-wait-state}.

\begin{figure}
%\jfigs{figures/ja05f11}{0.7}
\jfigs{figures/fsm3.eps}{\textwidth}
\caption{\label{fsm3}Closing after the other end closes.}
\end{figure}
The output of \netstat\ at the right end of Figure~\ref{fsm2} 
includes:
\begin{verbatim}
Active Internet connections
Proto Recv-Q Send-Q Local Address    Foreign Address      State      
tcp        0      0 A.B.C.D:P        W.X.Y.Z:Q            TIME_WAIT
\end{verbatim}

Figure~\ref{fsm3} shows the simpler sequence of events at the endpoint
that does not close first.  When the closing handshake message
arrives, an acknowledgment is sent immediately, and the connection
state becomes ``Close-Wait.''
The output of \netstat\ on this host shows:
\begin{verbatim}
Active Internet connections
Proto Recv-Q Send-Q Local Address    Foreign Address      State      
tcp        0      0 W.X.Y.Z:Q        A.B.C.D:P            CLOSE_WAIT
\end{verbatim}
%
At this point, it's all over: the implementation is just
waiting for the application to call \fcn{close()}.
When it does, the socket descriptor is deallocated and the final close
handshake is initiated. When it completes, the underlying socket structure is 
deallocated.

Although most applications use \fcn{close()}, \fcn{shutdown()}
actually provides more flexibility.  A call to \fcn{close()}
terminates \emph{both\/} directions of transfer and causes the file
descriptor associated with the socket to be deallocated.
Any undelivered data remaining in \rque\ is
discarded, and the flow control mechanism prevents any further
transfer of data from the other end's \sque.  All trace of the socket
disappears from the calling program.
Underneath, however, the associated socket structure continues to exist
until the other end initiates its closing handshake.
A program calling \fcn{shutdown()} with second argument
\const{SHUT\_WR} can continue to receive data on
the socket; only sending is prohibited.
The fact that the other end of the connection has closed is
indicated by \fcn{recv()} returning $0$ (once \rque\ 
is empty, of course)
to indicate that there will be no more data available on the connection.

In view of the fact that both \fcn{close()} and
\fcn{shutdown()} return without
waiting for the closing handshake to complete, you may wonder how the
sender can be assured that sent data has actually made it to the
receiving program (i.e., to \emph{Delivered}).  In fact, it is
possible for an application to call
\fcn{close()} or
\fcn{shutdown()} and have it
complete successfully (i.e., not return $-1$) \emph{while there
is still data in \sque}.  If either end of the connection then crashes
before the data makes it to \rque, data may be lost without the
sending application knowing about~it!

\callout{The best solution is to design the application protocol so that
whichever side closes first, does so
\emph{only after} receiving application-level assurance that its data
was received.}  For example, when our \file{TCPEchoClient.c} program
receives the same number of bytes as it sent, there should be nothing
more in transit in either direction, so it is safe for it to close the
connection.  (Note that there is no \emph{guarantee\/} that the bytes
received were those sent; the client is assuming that the server
implements the ``echo'' protocol.  In a real application the client
should certainly \emph{not\/} trust the server to ``do the right thing''.

The other solution is to modify the semantics of \fcn{close()}
by setting the \const{SO\_LINGER} socket option before
calling it.  The \const{SO\_LINGER} option specifies an amount of time
for the TCP
implementation to wait for the closing handshake to complete.  The
setting of \const{SO\_LINGER} and the specification
of the wait time is given to \fcn{setsockopt()} using the
\type{linger} structure:

\begin{inlinecode}
struct linger {
    int  l_onoff;   // Nonzero to linger
    int  l_linger;  // Time (secs.) to linger
};
\end{inlinecode}

\noindent To use the linger behavior, set \param{l\_onoff} to a
nonzero value and specify the time to linger in \param{l\_linger}.
When \const{SO\_LINGER} is set,
\fcn{close()} blocks \emph{until the
closing handshake is completed\/} or until the specified amount of time
passes.  If the handshake does not complete in time,
an error indication (\const{ETIMEDOUT}) is returned.
Thus, if \const{SO\_LINGER} is set and
\fcn{close()} returns no error, the
application is assured that everything it sent reached \rque.

At the
time of this writing, however, \fcn{close()} provides no indication
that the closing handshake failed to complete, even if the time limit
set by \fcn{setSoLinger} expires before the closing sequence
completes.  In other words, \fcn{setSoLinger} does not provide any
additional assurance to the application in current implementations.

The final subtlety of closing a TCP connection revolves around the
need for the Time-Wait state\label{time-wait-state}.  The TCP
specification requires that when a connection terminates, at least one
of the sockets persists in the Time-Wait state for a period of time
after both closing handshakes complete.  This requirement is motivated
by the possibility of messages being delayed in the network.  If both
ends' underlying structures go away as soon as both closing handshakes
complete, and a \emph{new} connection is immediately established
between the same pair of socket addresses, a message from the previous
connection, which happened to be delayed in the network, could arrive
just after the new connection is established.  Because it would
contain the same source and destination addresses, the old message
could be mistaken for a message belonging to the new connection, and
its data might (incorrectly) be delivered to the application.

Unlikely though this scenario may be, TCP employs multiple mechanisms
to prevent it, including the Time-Wait\ state.  The Time-Wait\ state
ensures that every TCP connection ends with a quiet time, during which
no data is sent.  The quiet time is supposed to be equal to twice the
maximum amount of time a packet can remain in the network.  Thus, by
the time a connection goes away completely (i.e., the socket structure
leaves the Time-Wait\ state and is deallocated) and clears the way for
a new connection between the same pair of addresses, no messages from
the old instance can still be in the network.  In practice, the length
of the quiet time is implementation dependent, because there is no
real mechanism that limits how long a packet can be delayed by the
network.  Values in use range from 4~minutes down to 30~seconds or
even shorter.

The most important consequence of Time-Wait\ is that as long as the
underlying socket structure exists, no other socket is permitted to
bind to the same local port.  (More on this below.)

\section{Demultiplexing Demystified}
\label{sect:demux}%

The fact that different sockets on the same machine can have the same
local address and port number is implicit in the discussions above.
For example, on a machine with only one IP address, every
new socket \fcn{accept()}ed via a listening socket
will have the same local address and port number
as the listening socket.  
Clearly the process
of deciding to which socket an incoming packet should be
delivered---that is, the \emph{demultiplexing} process---involves
looking at {more} than just the packet's destination address and port.
Otherwise there could be ambiguity about which socket an incoming
packet is intended for.  The process of matching an incoming packet to
a socket is actually the same for both TCP and UDP, and can be
summarized by the following points:
\begin{itemize}
\item The local port
in the socket structure \emph{must} match the destination port number
in the incoming packet.
\item Any address fields in the socket
structure that contain the wildcard value (*) are considered to match
\emph{any} value in the corresponding field in the packet.
\item If there is more than one socket structure that matches an incoming
packet for all four address fields, the one that matches using the
fewest wildcards gets the packet.
\end{itemize}

For example, consider a host with two IP addresses, 10.1.2.3 and
192.168.3.2, and with a subset of its active TCP socket structures, as
shown in Figure~\ref{demuxex}.  The structure labeled 0 is associated
with a listening socket and has port 99
with a wildcard local address.  Socket structure~1 is also for a
listening socket on the same port, but
with the local IP address 10.1.2.3 specified (so it will only accept
connection requests to that address).  Structure~2 is for a connection
that was accepted via structure 0's listening socket, and
thus has the same local port number (99), but also has its local and remote
Internet addresses filled in.  Other sockets belong to other active
connections.  Now consider a packet with source IP address
172.16.1.10, source port 56789, destination IP address 10.1.2.3, and
destination port 99.  It will be delivered to the socket associated
with structure 1, because that one matches with the fewest wildcards.

\begin{figure}
%\jfigs{figures/ja05f12}{0.7}
\jfigs{figures/demuxex.eps}{\textwidth}
\caption{\label{demuxex}Demultiplexing with multiple matching sockets.}
\end{figure}

When a program attempts to \fcnsys{bind()} to a particular local port
number, the existing sockets are checked to make sure that no socket
is already using that local port.  The call to \fcn{bind()} will fail
and set \const{EADDRINUSE}  if \emph{any} socket matches the local
port and local IP address (if any) specified in the argument to \fcn{bind()}.
This can cause problems in the following scenario:
%
\begin{enumerate}

\item A server's listening socket is bound to some particular port
  $P$.

\item The server accepts a connection from a client, which enters the
  Established state.

\item The server terminates for some reason---say, because the
  programmer has created a new version and wants to test it.
  When the server program exits,
  the underlying system automatically (and virtually) calls
 \fcn{close()} on  all of its existing sockets.  The socket that was
  in the Established state immediatey  transitions to the Time-Wait state.

\item
The programmer starts up a new instance of the server, which attempts
to \fcn{bind()} to port $P$.
\end{enumerate}
Unfortunately the new server's call to \fcnsys{bind()} will fail with
EADDRINUSE because of the old socket in Time-Wait state.

As of this writing, there are two ways around this.  One
is to wait until the underlying structure leaves the
Time-Wait\ state.  The other is for the server to set the
\const{SO\_REUSEADDR} socket option  \emph{before\/} calling
\fcn{bind()}. That lets \fcn{bind()} succeed in spite of the existence
of any sockets representing earlier connections to the server's port.
%
There is no danger of ambiguity, because the existing connections
(whether still in the Established or Time-Wait\ state) have remote
addresses filled in, while the socket being bound does not.
In general, the \const{SO\_REUSEADDR} option also
enables a socket
to bind to a local port to which another socket is already bound,
provided that the IP address to which it is
being bound (typically the wildcard \const{INADDR\_ANY}
address)
is different from the existing socket's.
The default \fcnsys{bind()} behavior is to disallow such requests.

\begin{exercises}

\item The TCP protocol is designed so that simultaneous connection
attempts will succeed.  That is, if an application using port P and
Internet address W.X.Y.Z attempts to connect to address A.B.C.D, port
Q, at the same time as an application using the same address and port
tries to connect to W.X.Y.Z, port P, they will end up connected to
each other.  Can this be made to happen when the programs use the
sockets API?

\item The first example of ``buffer deadlock'' in this chapter
involves the programs on {both} ends of a connection trying to send
large messages.  However, this is not necessary for deadlock.  How
could the \file{TCPEchoClient} from earlier chapters be made
to deadlock when 
it connects to the \file{TCPEchoServer} from that chapter?

\end{exercises}
