\chapter{Using UDP Sockets}
\label{chap:udp}%

The User Datagram Protocol (UDP) provides a simpler end-to-end
service than that of TCP.  In fact, UDP performs only two functions:
(1)~It adds another
layer of addressing (ports) to that of IP; and (2)~It detects data
corruption that may occur in transit and discards any corrupted
datagrams.
Because of this simplicity, UDP (datagram) sockets have some different
characteristics from the TCP (stream) sockets we saw earlier.

For example, UDP sockets do not have to be connected before being
used.  Where TCP is analogous to telephone communication, UDP is
analogous to communicating by mail: You do not have to ``connect''
before you send a package or letter, but you do have to specify the
destination address for each one.  In receiving, a UDP
socket is like a mailbox into which letters or packages from many
different sources can be placed.

Another difference between UDP sockets and TCP sockets is the way
they deal with message boundaries: UDP sockets preserve them.  This
makes receiving an application message simpler, in some ways, than with
TCP sockets.  We will discuss this further in
Section~\ref{sect:udpsendrcv}.
A final difference is that the end-to-end transport service UDP
provides is best effort: There is no guarantee that a message sent
via a UDP socket will arrive at its destination.  This means that a
program using UDP sockets must be prepared to deal with loss and
reordering of messages; we'll see an example of that later.

Again we introduce the UDP portion of the Sockets API through simple
client and server programs.  As before, they implement a trivial
echo protocol.  Afterward, we describe the API functionality in more
detail in Sections~\ref{sect:udpsendrcv} and~\ref{sect:udpconnect}.

\section{UDP Client}

\noindent Our UDP echo client, \file{UDPEchoClient.c}, looks similar to
our address-family-independent
\file{TCPEchoClient.c} in the way it sets up the server address
and communicates with the servier.  However,
it does not call \fcnrefsys{connect()}, it uses \fcnrefsys{sendto()}
and
\fcnrefsys{recvfrom()} instead of \fcnrefsys{send()} and
\fcnrefsys{recv()}, and it only needs
to do a single receive because UDP sockets preserve message
boundaries, unlike TCP's byte-stream service.
%
Of course, a UDP client only communicates with a UDP server.  Many
systems include a UDP echo server for debugging and testing
purposes; the server simply echoes whatever messages it receives back to
wherever they came from.
After setting up, our echo client performs the following steps: (1)~it
sends the echo string to the server, (2)~it receives the echo, and
(3)~it shuts down the program.

\jcode{UDPEchoClient.c}{code/UDPEchoClient.c}{1}{1}

\begin{topcode}

\tlcitems{Program setup and parameter parsing}{1--27}

The server address/name and string to echo are passed in as the first two
parameters.  We restrict the size of our echo message; therefore, we must
verify that the given string satisfies this restriction.  
Optionally, the client takes the server port or service name as the third
parameter. If no port is provided, the client uses the well-known echo protocol
service name, ``echo''.

\tlcitems{Get foreign address for server}{29--40}

For the server, we may be given an IPv4 address, IPv6 address,
or name to resolve.  For the optional port, we may be given a port number or
service name.  We use \fcnrefsys{getaddrinfo()} to determine the
corresponding address information (i.e., family, address, and port number). 
Note that we'll accept an address for any family (\constsys{af\_unspec}) for
UDP (\constsys{sock\_dgram} and \constsys{ipproto\_udp}); specifying
the latter two effectively restricts the returned address families
to IPv4 and IPv6. 
Note that \fcnrefsys{getaddrinfo()} may return multiple addresses;
by simply using the first, we may fail to communicate with the server
when communication is possible using an address later in the list.
\textbf{A production client should be prepared to try all returned addresses.}

\tlcitems{Socket creation and setup}{42--46}

This is almost identical to the TCP echo client, except that we
create a datagram socket using UDP.  Note that we do not need to
\fcnrefsys{connect()} before communicating with the server.

\tlcitems{Send a single echo datagram}{48--55}

With UDP we simply tell \fcnrefsys{sendto()} the datagram
destination.  If we wanted to,
we could call \fcnrefsys{sendto()} multiple times,
changing the destination on every call, thus communicating with
multiple servers through the same socket.
The first call to \fcnrefsys{sendto()} also assigns an
arbitrarily chosen local port number, not in use by any other socket, to the
socket identified by \var{sock}, because we have not bound the socket
to a port number.  We do not know (or care) what the chosen port number
is, but the server will use it to send the echoed message back to us.

\tlcitems{Get and print echo reply}{57--76}

\begin{bottomcode}

\blcitems{Receive a message}{58--68}

We initialize \var{fromAddrLen} to contain the
size of the address buffer (\var{fromAddr}) and then
pass its address  as the  last parameter.
\fcnrefsys{recvfrom()} blocks until a UDP datagram addressed to
this socket's port arrives.  It
then copies the data from the first arriving datagram
into \var{buffer} and copies the
Internet address and (UDP) port number of its source from the
packet's headers into the structure \var{fromAddr}.
Note that the data buffer is actually one byte bigger than
\const{maxstringlength}, to allow us to add a null byte to terminate the
string.

\blcitems{Check message source}{70--75}

Because there is no connection, a received message can come from
any source.  The output parameter \var{fromAddr} informs us of
the datagram's source, and we check it to make sure it matches the
server's Internet address.  We use our own function,
\fcnref{SockAddrsEqual()}, to
perform protocol-independent comparison of socket addresses.  Although it is
very unlikely that a packet would ever arrive from any other source, we include
this check to emphasize that it is possible.
There's one other complication.
For applications with multiple or repeated requests, we must keep 
in mind that UDP messages may be reordered and arbitrarily delayed so
simply checking 
the source address and port
may not be sufficient.  (For example, the DNS protocol over UDP
uses an identifier field to link requests and responses and detect
duplication.)  This is our last use of the address returned
from \fcnrefsys{getaddrinfo()} so we can free the associated storage.

\blcitems{Print received string}{77--78}

Before printing the received data as a string, we
first ensure that it is null-terminated.

\end{bottomcode}

\tlcitems{Wrap up}{80--81}

\end{topcode}

This example client is fine as an introduction to the UDP socket
calls; it will work correctly most of the time.  However,
\callout{it would not
be suitable for production use, because if a message is lost going
to or from the server, the call to \fcnrefsys{recvfrom()} blocks
forever, and the program does not terminate.}
Clients generally deal with this problem through the use of
\emph{timeouts}, a subject we cover later in Section~\ref{sect:timeouts}.

\section{UDP Server}

\noindent Our next example program implements the UDP version of the echo
server, \file{UDPEchoServer.c}.  The server is very simple: It loops forever,
receiving a message and then sending the same message back to wherever it came
from.  Actually, the server only receives and
sends back the first 255 characters of the message; any excess is
silently discarded by the sockets implementation.  (See
Section~\ref{sect:udpsendrcv} for an explanation.)

\jcode{UDPEchoServer.c}{code/UDPEchoServer.c}{1}{1}

\begin{topcode}

\tlcitems{Program setup and parameter parsing}{1--15}

\tlcitems{Parse/resolve address/port args}{17--29}

The port may be specified on the command line as a port number or service name.
We use \fcnrefsys{getaddrinfo()} to determine the actual local port
number.  As with our UDP client, we'll accept an address for any family
(\constsys{af\_unspec}) for UDP (\constsys{sock\_dgram} and
\constsys{ipproto\_udp}).  We want our UDP server to
accept echo requests from any of its interfaces.  By
specifying the \constsys{ai\_passive} flag, \fcnrefsys{getaddrinfo()} returns
the wildcard Internet address (e.g., \constsys{inaddr\_any} for IPv4 or
\constsys{in6addr\_any} for IPv6).  \fcnrefsys{getaddrinfo()} may return
multiple addresses; we simply use the first.

\tlcitems{Socket creation and setup}{30--35}

This is nearly identical to the TCP echo server, except that we
create a datagram socket using UDP.  Also, we do not need to
call
\fcnrefsys{listen()} because there is no connection setup---the
socket is ready to receive messages as soon as it has an address.

\tlcitems{Iteratively handle incoming echo requests}{43--59}

Several key differences between UDP and TCP servers are demonstrated
in how each communicates with the client.  In the TCP server, we
blocked on a call to \fcnrefsys{accept()} awaiting a connection
from a client.  Since UDP servers do not establish a
connection, we do not need to get a new socket for each client.
Instead, we can immediately call \fcnrefsys{recvfrom()}
with the same socket that was bound to
the desired port number.

\begin{bottomcode}

\blcitems{Receive an echo request}{46--51}

\fcnrefsys{recvfrom()} blocks until a datagram is received from a
client.  Since there is no connection, each datagram may come from
a different sender, and we learn the source at the same time
we receive the datagram. \fcnrefsys{recvfrom()} puts
the address of the source in \var{echoClntAddr}.  The length
of this address buffer is specified by \var{cliAddrLen}.

\blcitems{Send echo reply}{56--58}

\fcnrefsys{sendto()} transmits the data in \var{echoBuffer} back
to the address specified by \var{clntAddr}.  Each received
datagram is considered a single client echo request,
so we only need a single send and receive---unlike the TCP echo server,
where we needed to receive until the client closed the connection.

\end{bottomcode}

\end{topcode}

\section{Sending and Receiving with UDP Sockets}
\label{sect:udpsendrcv}%

As soon as it is created, a UDP socket can be used to send/receive
messages to/from any address and to/from many \emph{different}
addresses in succession.  To allow the
destination address to be specified for each message,
the sockets API provides a different sending routine
that is generally used with UDP sockets: \fcnrefsys{sendto()}.
Similarly, the \fcnrefsys{recvfrom()} routine returns the
source address of each received message in addition to the message
itself.
\begin{inlinefcn}
\type{ssize\_t} \fcnrefsys{sendto}(\type{int} \param{socket},
\type{const void *}\param{msg}, \type{size\_t} \param{msgLength}, 
\type{int} \param{flags},\\
\hspace*{2in} \type{const struct sockaddr *}\param{destAddr}, 
\type{socklen\_t} \param{addrLen}) \\
\type{ssize\_t} \fcnrefsys{recvfrom}(\type{int} \param{socket},
\type{void *}\param{msg}, \type{size\_t} \param{msgLength}, 
\type{int} \param{flags},\\
\hspace*{2in} \type{struct sockaddr *}\param{srcAddr}, 
\type{socklen\_t *}\param{addrLen})
\end{inlinefcn}

The first four parameters to \fcnrefsys{sendto()} are the same
as those
for \fcnrefsys{send()}.  The two additional parameters specify the
message's destination.  Again, they will
invariably be a pointer to a \typesys{struct sockaddr\_in} and
its size, respectively, or a pointer to
a \typesys{struct sockaddr\_in6} and its size, respectively.
%
Similarly, \fcnrefsys{recvfrom()} takes the same  parameters  as
\fcnrefsys{recv()} but, in addition, has two parameters that inform
the caller of the source of the received datagram.
One thing to note is that \param{addrLen} is an
\emph{in-out} parameter in \fcnrefsys{recvfrom()}:
On input it specifies the size of the
address buffer \param{srcAddr}, which will typically be a
\typesys{struct sockaddr\_storage} in IP-version-independent code.
On output, it specifies the size of the
address that was actually copied into the buffer.
`\callout{Two errors often made by
novices are (1)~passing an integer value instead of a pointer to an
integer for \param{addrLen}
and (2)~forgetting to initialize the pointed-to length variable
to contain the appropriate size.}

We have already pointed out a
subtle but important difference between TCP and UDP, namely that
\emph{UDP preserves message boundaries}.
In particular, each call to \fcnrefsys{recvfrom()}
returns data from at most one \fcnrefsys{sendto()} call.
Moreover, different calls
to \fcnrefsys{recvfrom()} will never return data from the same
call to \fcnrefsys{sendto()}
(unless you use the msg\_peek
flag with \fcnrefsys{recvfrom()}---see next page).

When a call to \fcnrefsys{send()} on a TCP socket returns, all the caller
knows is that the data has been copied into a buffer for transmission; the
data may or may not have actually been transmitted yet. (This is explained
in more detail in Chapter~\ref{chap:under}.)
However, UDP does not buffer data for possible retransmission because
it does not recover from errors.
This means that by the time a call to \fcnrefsys{sendto()} on a
UDP socket returns, the message has been passed to the underlying
channel for transmission and is (or soon will be) on its way out the
door.

Between the time a message arrives from the network and the time its
data is returned via \fcnrefsys{recv()} or
\fcnrefsys{recvfrom()}, the
data is stored in a first-in, first-out (FIFO) receive buffer.  With a 
connected TCP socket, all
received-but-not-yet-delivered bytes are treated as one continuous sequence
(see Section~\ref{sect:sendrec}).  For a UDP socket,
however, the bytes from different messages may have come from
different senders.  Therefore, the boundaries between them need to be
preserved so that the data from each message can be returned with the
proper address.  The buffer really contains a FIFO sequence of ``chunks''
of data, each with an associated source address.  A call to
\fcnrefsys{recvfrom()} will never return more than one of these
chunks.  However, if \fcnrefsys{recvfrom()} is called with size parameter
$n$, and the size of the first chunk in the receive FIFO is bigger than
$n$, only the first $n$ bytes of the chunk are returned.  \callout{The
remaining bytes are quietly discarded, with no indication to the
receiving program}.

For this reason, a receiver should always supply a buffer big enough
to hold the largest message allowed by its application protocol at the
time it calls
\fcnrefsys{recvfrom()}. This technique will guarantee that no data will be
lost.  The maximum amount of data that can ever be returned by
\fcnrefsys{recvfrom()} on a UDP socket is 65,507 bytes---the
largest payload that can be carried in a UDP datagram.
% XXXXXX THIS CHANGEs WITH IPv6!?
% IP header is bigger, so ...

Alternatively, the receiver can use the \const{msg\_peek}
flag with \fcnrefsys{recvfrom()} to
``peek'' at the first chunk waiting to be received.
This flag causes the received data to remain in the
socket's receive FIFO so it can be received more than once.
This strategy can be
useful if memory is scarce, application messages vary widely in
size, and each message carries information about its size in the first
few bytes.  The receiver first calls \fcnrefsys{recvfrom()} with
\const{msg\_peek} and a small buffer, examines the first
few bytes of the message to determine its size, and then calls
\fcnrefsys{recvfrom()} again (without \const{msg\_peek})
with a buffer big enough to hold the entire message.  In the
usual case where memory is not scarce, using a buffer big enough for
the largest possible message is simpler.


\section{Connecting a UDP Socket}
\label{sect:udpconnect}
It is possible to
call \fcnrefsys{connect()} on a UDP socket to fix the
destination address of future datagrams sent over the socket.
Once connected, you may use
\fcnrefsys{send()} instead of \fcnrefsys{sendto()} to transmit datagrams
because you no longer need to specify the destination address.
%
In a similar way you may use \fcnrefsys{recv()} instead of
\fcnrefsys{recvfrom()}, because a connected UDP socket can \emph{only\/}
receive datagrams from the associated foreign address and port, so
after calling \fcnsys{connect()} you
know the source address of any incoming datagrams.
In fact, after
connecting, you may \emph{only} send and receive to/from the address
specified to \fcnrefsys{connect()}.  Note that connecting and then using \fcnrefsys{send()} and
\fcnrefsys{recv()} with UDP does not change how UDP behaves.  Message 
boundaries are still preserved, datagrams can be lost, etc.  You can ``disconnect'' by calling
\fcnrefsys{connect()} with an address family of \constsys{AF\_UNSPEC}.

Another subtle advantage to calling \fcnrefsys{connect()} on a UDP
socket is that it 
enables you to receive error indications that result from earlier actions
on the socket.  The canonical example is sending a datagram to a
non-existent server or port.  When this happens, the \fcn{send()} that
eventually leads to the error returns with no indication of error.
Some time later, an error message is delivered to your host,
indicating that the sent datagram encountered a problem.
Because this datagram is a \emph{control\/} message and not a regular
UDP datagram, the system can't always tell where to send it if your
socket is unconnected, because an unconnected socket has no
associated foreign address and port.
However, if your socket is connected, the system is able to match the
information in the error datagram with your socket's associated foreign IP
address and port. (See Section~\ref{sect:demux} in for details about this process.)
Note such a control error message being delivered
to your socket will result in
an error return from a \emph{subsequent\/}
system call (for example the
\fcnsys{recv()} that was intended to get the reply)
not the offending \fcnrefsys{send()}.

\begin{exercises}

\item Modify \file{UDPEchoClient.c} to use \fcnrefsys{connect()}.
After the final \fcnrefsys{recv()}, show how to disconnect 
the UDP socket.  Using 
\fcnrefsys{getsockname()} and \fcnrefsys{getpeername()}, print the local and 
foreign address before and after \fcnrefsys{connect()}, and after disconnect.

\item Modify \file{UDPEchoServer.c} to use \fcnrefsys{connect()}.

\item Verify experimentally the size of the largest datagram you can send
and receive using a UDP socket.  Is the answer different for IPv4 and IPv6?

\item While \file{UDPEchoServer.c} explicitly specifies its local port
number using \fcnrefsys{bind()}, we do not call \fcnrefsys{bind()} in
\file{UDPEchoClient.c}. How is the UDP echo client's socket given a port
number? Note that the answer is different for UDP and TCP.  We can select the
client's local port using \fcnrefsys{bind()}.  What difficulties might we
encounter if we do this?

\item Modify \file{UDPEchoClient.c} and \file{UDPEchoServer.c} to
allow the largest echo string possible where an echo request is
restricted to a single datagram.

\item Modify \file{UDPEchoClient.c} and \file{UDPEchoServer.c} to
allow arbitrarily large echo strings. You may ignore datagram loss and
reordering (for now).

\item Using \fcnrefsys{getsockname()} and \fcnrefsys{getpeername()},
modify \file{UDPEchoClient.c} to print the local and foreign address
for the server socket immediately before and after
\fcnrefsys{sendto()}.

\item You can use the same UDP socket to send datagrams to many different
destinations.  Modify \file{UDPEchoClient.c} to send and receive an echo
datagram to/from two different UDP echo servers.  You can use the book's server
running on multiple hosts or twice on the same host with different ports.

\end{exercises}

