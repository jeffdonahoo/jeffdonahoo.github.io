% Shortcut for \noindent
%\newcommand{\noind}{\noindent}

% Macro to introduce invisible (play) character as formatting trick
%\newcommand{\pc}{\mbox{}}

% Macro to define appearance of program source code file name in text
\newcommand{\codename}[1]{\texttt{#1}}

% Macro to define appearance of a term to define
\newcommand{\defn}[1]{\index{#1}\emph{#1}}

% Macro to input code from a source file and label
%
% 1 - Label of code segment
% 2 - source file
%
% Example:  \jcode{Bob}{bob.c}
%
% Bob
% ____________________________________________________________
% #include <stdio.h>...
% \newcommand{\jcode}[2]{\label{#2}\vspace*{0.05in}\hspace*{\fill}{\bf \codename{#1}} \\
% \hrule \small{\listinginput{0}{#2}} \hrulefill\\ \hspace*{\fill}{\bf \codename{#1}}}
% I hope none of our programs are over 10000 lines long!
% 1 - Last line number from file (optional) - SPECIFY IN [] when using
% 2 - Title of code segment
% 3 - source file
% 4 - First line number to display
% 5 - First line number from file (start from 1)
%
% Example:  \jcode{Bob}{bob.c}{0}{1}
%           \jcode[20]{Bob}{bob.c}{0}{1}
\renewcommand{\theFancyVerbLine}{\arabic{FancyVerbLine}}
% 1 - Title of code segment
% 2 - source file
\newcommand{\jcodestartdivide}[2]{\index{#2}\label{#2}\vspace*{0.05in}\hspace*{\fill}\textbf{\codename{#1}} \\ \hrule}
% 1 - Title of code segment
\newcommand{\jcodestopdivide}[1]{\hrulefill \\ \hspace*{\fill}\textbf{\codename{#1}}}
% 1 - Last line number from file (optional) - SPECIFY IN [] when using
% 2 - source file
% 3 - First line number to display
% 4 - First line number from file (start from 1)
%\newcommand{\jcodenoem}[4][10000]{\VerbatimInput[numbers=left,fontfamily=courier,fontsize=\small,firstnumber=#3,
\newcommand{\jcodenoem}[4][10000]{\VerbatimInput[numbers=left,fontfamily=courier,fontsize=\normalsize,firstnumber=#3,
firstline=#4,lastline=#1,xleftmargin=0.23in,numbersep=5pt]{#2}}
% 1 - Last line number from file (optional) - SPECIFY IN [] when using
% 2 - source file
% 3 - First line number to display
% 4 - First line number from file (start from 1)
\newcommand{\jcodeem}[4][10000]{\VerbatimInput[numbers=left,fontfamily=courier,fontsize=\normalsize,firstnumber=#3,
firstline=#4,lastline=#1,xleftmargin=0.23in,numbersep=5pt,fontseries=b]{#2}}
\newcommand{\jcode}[5][10000]{\jcodestartdivide{#2}{#3} \jcodenoem[#1]{#3}{#4}{#5} \jcodestopdivide{#2}}

% Use this to connect previous code snippets
%
% Example:  This prints the start divider, the first 5 lines normally, the remaining lines emphasized (bold)
%           and the stop divider
% \jcodestartdivide{Bob}{bob.c}
% \jcodenoem[5]{bob.c}{0}{1}
% \jcodeconnect
% \jcodeem{bob.c}{0}{6}
% \jcodestopdivide{Bob}
\newcommand{\jcodeconnect}{\vspace*{-0.27in}}

\newcommand{\jtitle}{JUNK}
\newcommand{\jfile}{JUNK}

% Macro to create pamphelet part.  Pamphlet consist of (in decreasing 
% order of importance): Parts, chapters (\pampchap), sections
% (\pampsect), and subsection (\pampsubsect, \apihdr)
%
% 1 - Part name (e.g. Tutorial)
%
% Example:  \pamppart{Tutorial}
%
% <new page>
% ...
%                    Part I:
% ...
%                   Tutorial
% ...
% \newcounter{partcnt}
% \renewcommand{\thepartcnt}{\Roman{partcnt}}
% \newcommand{\pamppart}[1]{
% \stepcounter{partcnt}
% \newpage % create new page
% \thispagestyle{empty}  % No page number

% \vspace*{\fill}

% \begin{center}  % Include pamphlet part major name on one line 
%                 % and minor name on next line
% {\bf\Huge Part \thepartcnt\\ \vskip .5in #1}
% \addtocontents{toc}{\vspace*{0.1in}\noindent{\LARGE\bf Part \Roman{partcnt}\hfill #1}\vspace*{0.1in}}
% \end{center}
% \vfill  % Fill remainder of page

% \newpage  % create new page
% }

\newcommand{\pamppart}[1]{\part{#1}}

% Macro to create chapter header (e.g. Introduction)
%
% 1 - Chapter name
%
% Example: \pampchap{Introduction}
%
% 1 Introduction (right justified)
%
% \newcounter{chapcnt} %[partcnt] Add back in to get chapcnt to restart with each part
% \renewcommand{\thechapcnt}{\arabic{chapcnt}}
% \newcommand{\pampchap}[1]{\stepcounter{chapcnt}\label{chap:#1}\vspace*{0.1in}\noindent{\Large\bf Chapter~\thechapcnt:  #1}\\[0.1in]\addcontentsline{toc}{section}{\thechapcnt~#1}}

%\newcommand{\pampchap}[1]{\newpage%
%\stepcounter{chapter}\vspace*{0.1in}\noindent{\LARGE\bf Chapter~\thechapter:  #1}
%\\[0.1in]\addcontentsline{toc}{chapter}{\protect\numberline{\thechapter}#1}}

\newcommand{\pampchap}[1]{\chapter{#1}}

% Macro to create section header (e.g. What is a socket)
%
% 1 - Section name
%
% Example: \pampsect{What is a socket}
%
% 1.1 What is a socket
% ___________________________________________________________
%
% \newcounter{sectcnt}[chapcnt]
% \renewcommand{\thesectcnt}{\arabic{chapcnt}.\arabic{sectcnt}}
% \newcommand{\pampsect}[1]{\stepcounter{sectcnt}\vspace*{0.1in}\noindent{\bf \thesectcnt~#1} \hrulefill\addcontentsline{toc}{subsection}{\thesectcnt~#1}}

\newcommand{\pampsect}[1]{\section{#1}}

% Macro to create subsection header (e.g. What is a socket)
%
% 1 - Subsection name
%
% Example: \pampsubsect{What is a socket}
%
% 1.1.1 What is a socket
%
% \newcounter{subsectcnt}[sectcnt]
% \renewcommand{\thesubsectcnt}{\arabic{chapcnt}.\arabic{sectcnt}.\arabic{subsectcnt}}
% \newcommand{\pampsubsect}[1]{\stepcounter{subsectcnt}\vspace*{0.1in}\noindent{\bf \thesubsectcnt~#1}\addcontentsline{toc}{subsection}{\thesubsectcnt~#1}}

\newcommand{\pampsubsect}[1]{\subsection{#1}}

% Macro to create header for an API chapter (e.g. Socket Creation)
%
% 1 - API chapter name
%
% Example: \apichap{Socket Creation}
%
%\newcommand{\apichap}[1]{\pagebreak[4]\label{api:#1}\vspace*{0.1in}%
\newcommand{\apichap}[1]{\vspace*{0.1in}%
\noindent{\Large\bf #1}\\[0.1in]\addcontentsline{toc}{chapter}{#1}}

% Macro to create header for API (e.g. socket())
%
% 1 - API function name
% 2 - Additional phrase
%
% Example: \apihdr{listen()}{- Stream/TCP socket only}
%
% o listen() - Stream/TCP socket only
% ----------
%
\newcommand{\apihdr}[2]{\noindent $\circ$ \texttt{\large \underline{#1}#2}%
\addcontentsline{toc}{section}{#1}}

% Macros for appearance of functions (\fnc), function types (\type), and 
% function parameters (\var)
\newcommand{\var}[1]{\textsl{#1}}
\newcommand{\varsys}[1]{\var{#1}\index{#1}}
\newcommand{\const}[1]{\textsc{#1}}
\newcommand{\constsys}[1]{\const{#1}\index{#1}}
\newcommand{\type}[1]{\textbf{#1}}
\newcommand{\typesys}[1]{\type{#1}\index{#1}}
\newcommand{\fcn}[1]{\texttt{#1}}
\newcommand{\fcnsys}[1]{\fcn{#1}\index{#1}}
\newcommand{\structmem}[1]{\texttt{#1}}
\newcommand{\param}[1]{\var{#1}}
\newcommand{\file}[1]{\texttt{#1}}
\newcommand{\fcnref}[1]{\texttt{#1}}
\newcommand{\fcnrefsys}[1]{\fcnref{#1}\index{#1}}
\newcommand{\returncode}[1]{$#1$}
\newcommand{\exec}[1]{\texttt{#1}}
\newcommand{\errnum}[1]{\constsys{#1}}
\newcommand{\errno}{\texttt{errno}}
\newcommand{\codesegref}[1]{\texttt{#1}}

% Macro for code file reference
\newcommand{\coderef}[1]{\file{#1} (pg.~\pageref{#1})}

% Macro for defining a parameter to an API function
\newcommand{\paramdesc}[2]{\var{#1} & --- & #2 \\}

% Macro for adding an include file
\newcommand{\newincl}[1]{\#include $<$#1$>$}

% Macro to place box around API entries to avoid page breaks
\newcommand{\apibox}[1]{\parbox{\textwidth}{#1}}

% Macro to define an API function
% 
% Parameters:
% 1 - function name (e.g. socket())
% 2 - option phrase (e.g. Stream/TCP sockets only)
% 3 - function description (e.g. creates a socket...)
% 4 - include files
% 5 - function declaration (e.g. socket(int ...))
% 6 - parameter table (e.g. \begin{tabular}...)
% 7 - return information (e.g. -1 for failure...)
%
\newcommand{\apifunc}[7]{
\index{#1}
\apihdr{#1}{#2}\\[0.1in]
%\vspace*{0.1in}
#3
\begin{quote}

#4	

#5

\begin{quote}
#6
\end{quote}
#7
\end{quote}
}

% No parameter version of \apifunc
\newcommand{\apifuncnp}[6]{
\index{#1}
\apihdr{#1}{#2}\\[0.1in]
%\vspace*{0.1in}
#3  #6
\begin{quote}

#4	

\fcn{#5}
\end{quote}
}

% Macro for code references in api specifications
%
% 1 - Name of program
% 2 - Program page reference
% 3 - Lines
\newcommand{\seecode}[3]{\noindent See \codename{#1} pg.~\pageref{#2}/ln. #3 \\}

% Top-level code description environment
\newenvironment{topcode}{\begin{enumerate}}{\end{enumerate}}

% Macro for top level code description
%
% Parameters:
% 1 - Tag
% 2 - Line numbers
%
\newcommand{\tlcitem}[2]{\item \textbf{#1:} line {#2}}
\newcommand{\tlcitems}[2]{\item \textbf{#1:} lines {#2}}

% Bottom-level code description environment
\newenvironment{bottomcode}{\begin{itemize}}{\end{itemize}}

% Macro for top level code description
%
% Parameters:
% 1 - Tag
% 2 - Line numbers
%
\newcommand{\blcitem}[2]{\item \textbf{#1:} line {#2}}%
\newcommand{\blcitems}[2]{\item \textbf{#1:} lines {#2}}%
%
% Constant for default client IP/name and server IP/name
\newcommand{\clientIPiv}{169.1.1.2}
\newcommand{\clientName}{tractor.farm.com}
\newcommand{\serverIPiv}{169.1.1.1}
\newcommand{\serverName}{base.farm.com}
\newcommand{\clientIPvi}{FE80::2}
\newcommand{\serverIPvi}{FE80::1}
\newcommand{\mkpweb}{\textit{www.mkp.com}}

% Termination character
\newcommand{\termchar}{0}

% Names for queues. XXXX  CHANGE THESE -- KLC
\newcommand{\sque}{\textit{SendQ}}
\newcommand{\rque}{\textit{RecvQ}}
\newcommand{\dque}{\textit{Delivered}}
\newcommand{\sqsize}{\textit{SQS}}
\newcommand{\rqsize}{\textit{RQS}}

% Environment for including shell execution
%
% Parameters:
% 1 - C
% 2 - Line numbers
%
\newcommand{\prompt}{\% }
\newcommand{\typed}[1]{\textbf{#1}}
\newcommand{\response}[1]{#1}
\newenvironment{shell}
{\begin{quote} \ttfamily \vskip -.1in}
{\end{quote}\vskip -.1in}

% EPS figure insertion
%
% Parameters:
% 1 - Filename
% 2 - Options (e.g. ,width=11in,height=12in)
%
% Example: \jfig{bob.eps}{}
%
\newcommand{\jfig}[2]{\centerline{\psfig{file=#1#2}}} %XXX don't use this!
\newcommand{\jfigs}[2]{\begin{center}\includegraphics[width={#2}]{#1}%
\end{center}}

% Inline code insertion
\DefineVerbatimEnvironment%
  {inlinecode}{Verbatim}
  {fontfamily=courier,fontsize=\normalsize,xleftmargin=.25in}

% Inline function definitions
\newenvironment{inlinefcn}
{\vspace*{0.2in}\noindent\rule{\textwidth}{1pt}\\[0.1in]}
{\\[0.1in]\noindent\rule{\textwidth}{1pt}\vspace*{0.1in}}

% Ken macros
\newcommand{\Fig}[1]{Figure~\ref{#1}}
\newcommand{\Sect}[1]{Section~\ref{#1}}

% List redefinitions
\newcounter{enumct}
\renewenvironment{enumerate}{\begin{list}%
{\arabic{enumct}. }%
{\usecounter{enumct}
\setlength{\parsep}{0.15ex plus0.1ex minus0.1ex}
\setlength{\topsep}{0.15ex plus0.1ex minus0.1ex}
\setlength{\partopsep}{0.15ex plus0.1ex minus0.1ex}
%\setlength{\leftmargin}{0in}
\setlength{\itemsep}{0.15ex plus0.1ex minus0.1ex}}}
{\end{list}}

\renewenvironment{itemize}{\begin{list}%
{$\bullet$}%
{\setlength{\parsep}{0.15ex plus0.1ex minus0.1ex}
\setlength{\topsep}{0.15ex plus0.1ex minus0.1ex}
\setlength{\partopsep}{0.15ex plus0.1ex minus0.1ex}
%\setlength{\leftmargin}{0in}
\setlength{\itemsep}{0.15ex plus0.1ex minus0.1ex}}}
{\end{list}}

\newcommand{\copyrightnotice}{%
\begin{center}
\copyright 2000, Morgan Kaufmann Publishers.  Reproduced with permission from 
\emph{The Pocket Socket Guide in C}
\end{center}}

% Thought question section -- XXXX DEPRECATED
\newenvironment{tquestlist}{\section*{Thought Questions}\begin{enumerate}}{\end{enumerate}}
% Exercises section
\newenvironment{exercises}{\section*{Exercises}\begin{enumerate}}{\end{enumerate}}
% Marginal mark for points to be emphasized
\newsavebox{\NB}
\newcommand{\callout}[1]{\textbf{#1}\marginpar{\usebox{\NB}}}
\sbox{\NB}{\Huge !*!}
