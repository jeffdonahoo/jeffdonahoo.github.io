% OUTLINE
% 1.1  IP Version Independence
%        TCPEchoClient/Server.c
% 1.2  Finding Information By Name
% 1.3  Advanced Name Resolution
% 1.4  Identifying Endpoints
%        getpeer/sockname() - Moved earlier?
	
\chapter{Of Names and Address Families}
\label{chap:addrindep}

At this point, you know enough to build working TCP clients and servers.
However, our examples so far, while useful enough, nevertheless have a
couple of 
features that could be improved.  First, \emph{the only way to specify a
destination is with an IP address},
like ``169.1.1.1'' or ``FE80:1034::2A97:1001:1''.
This is a bit painful for most humans, who are---let's face it---not
that good at dealing with long strings of numbers that have to be
formatted just right.
That's why most applications allow the use of \emph{names\/}
like ``www.mkp.com'' and ``server.example.com'' to specify
destinations, in addition to Internet Addresses.
%
But the Sockets API, as we've seen, \emph{only\/} takes numerical
arguments so applications need a way to convert names to
the required numerical form.

Another problem with our examples so far is that \emph{the choice whether
to use IPv4 or IPv6 is wired into the code}---each progam we've seen
deals with only \emph{one\/} version of the IP protocol.  That was
by design---to keep things simple.  But wouldn't it be better
if we could \emph{hide\/} this choice from the rest of the
code, letting the argument(s) determine whether to a socket for IPv4
or IPv6 is created?

It turns out that the API provides solutions to
both of these problems---and more!
In this chapter we'll see how to
(i)~access the \emph{name service\/} to convert between names and
numeric quantities; and (ii)~write code that
chooses between IPv4 and IPv6 at run time.
%
%Before we proceed, however, note that we'll be dealing with
%a rather complex interface.  We'll therefore introduce its use gradually,
%ignoring some of its functionality (and input parameters) at first.

% old text...
%% you have some tough decisions to make.  One major decision is IPv4
%% vs. IPv6.  As we write this book, IPv4 is still the dominant protocol
%% of the Internet; however, as we pointed out in Chapter~\ref{chap:intro},
%% IPv6 is making inroads and most hosts are already IPv6-ready.  Even
%% after you've resolved your IP version question, what about address and
%% port?  As an Internet end-user, you probably rarely, if ever, use
%% addresses directly; instead, you refer to names (e.g., google.com).
%% As for ports, most Internet end-users have never heard of them.  In
%% this chapter, we'll explore mechanisms to create clients and servers
%% that are oblivious to IP protocol version and even IP addresses and
%% ports.

\section{Mapping Names to Numbers}

% XXX Kill this?
Identifying endpoints with strings of dot- or
colon-separated numbers is not very user-friendly, but that's not the
only reason to prefer names over addresses.
Another is that a host's Internet address is tied to the part of the
network to which it is connected.  This is a source of inflexibility:
If a host moves to another network or changes Internet service
providers (ISPs), its Internet address generally has to change.
Then everybody who refers to the host by that address has to
be informed of the change, or they won't be able to access the host!
When this book was written, the Web server for the
publisher of this text, Morgan Kaufmann, had an Internet address of
129.35.69.7.  However, we invariably refer to that Web server as \mkpweb.
Obviously, \mkpweb\ is easier to remember than 129.35.69.7.
In fact, most likely this is how you typically think of specifying a
host on the Internet, by name.  In addition, if Morgan Kaufmann's Web
server changes its Internet address for some reason (e.g., new ISP,
server moves to another machine), simply changing the mapping of
www.mkp.com from 129.35.69.7 to the new Internet address
allows the change to be transparent to all
programs that use the name to identify the Web server\footnote{MK's 
address was actually 208.164.121.48 when we wrote the first edition.
Presumably they changed their address to help us make this point.}.
%One final
%advantage of using names is ability to map a single name to multiple addresses
%as one way to improve service scalability and robustness.
%As your web service becomes more and more popular, you can distribute
%it across multiple servers.  When an end user application accesses the service
%by name, it can randomly select one of the possible addresses, providing
%a primitive form of load distribution.  In addition, if one of the servers
%fail or the network is partitioned, the end user application can simply try
%other addresses in the list until one works.

To solve these problems, most implementations of the sockets API
provide access to a \emph{name service} that maps names to other
information, including Internet addresses.  You've already seen names
that map to Internet addresses (``www.mkp.com'').  Names for services
(e.g., ``echo'') can also be mapped to port numbers.
%
The process of mapping a name to a numeric quantity (address or port
number) is called \emph{resolution}.  There are a number of ways to
resolve names into binary quantities; your system probably provides
access to several of these.  Some of them involve interaction with
other systems ``under the covers''; others are strictly local.

It is critical to remember that \callout{a name service is not required for
TCP/IP to work}.  Names simply provide a level of indirection, for the
reasons discussed above.  The host naming service can
access information from a wide variety of sources.  Two of the primary
sources are the \emph{Domain Name System\/} (DNS) and local configuration
databases.  The DNS~\cite{RFC1034} is a distributed database
that maps \emph{domain names} such as ``www.mkp.com''
to Internet addresses and other information; the
DNS protocol~\cite{RFC1035} allows hosts connected to the Internet
to retrieve information from that database using TCP or UDP.
Local configuration databases are generally operating-system-specific
mechanisms for name-to-Internet-address mappings.
Fortunately for the programmer, the details of how the name
service is \emph{implemented\/} are hidden behind the API, so
the only thing we need to know is how to ask it to \emph{resolve\/} a
name. 

\subsection{Accessing the Name Service}

The preferred interface to the
name service interface is through the function
\fcnsys{getaddrinfo()}:\footnote{Historically, other functions
were available for this purpose, and many applications still use them.
However, they have several shortcomings and are considered
obsolescent as of the POSIX 2001 standard.}
%

\begin{inlinefcn}
\type{int} \fcnsys{getaddrinfo} (\type{const char *}\param{hostStr},
\type{const char *}\param{serviceStr},\\
\hspace*{1.1in} \type{const struct addrinfo *}\param{hints},
\type{struct addrinfo **}\param{results});
\end{inlinefcn}

The first two arguments to \fcn{getaddrinfo()} point to
null-terminated character strings representing a host name or address and a
service name or port number, respectively.
The third argument describes the kind of information to be
returned; we discuss it below.
The last argument is the location of a \type{struct addrinfo} pointer,
where a pointer to a linked list containing results will be stored.
The return value of \fcn{getaddrinfo()} indicates whether the
resolution was
successful (0) or unsuccessful (nonzero error code).

Using \fcn{getaddrinfo()} entails using two other auxiliary functions:

\begin{inlinefcn}
%
\type{void }\fcnrefsys{freeaddrinfo(\typesys{struct addrinfo
    *}\param{addrList})};\\
%
\type{const char *}\fcnrefsys{gai\_strerror(\type{int} \param{errorCode})};
\end{inlinefcn}

\noindent \fcnsys{getaddrinfo()} creates a dynamically allocated linked list of 
results, which must be deallocated but only after the caller is
finished with the list.  Given the pointer to the head of the result list, 
\fcnsys{freeaddrinfo()} frees all the storage
allocated for the list.  Failure to call this method
can result in a pernicious memory leak.  The method \callout{should only be
called when the program is finished with the returned
information}; no information contained in the list of results is
reliable after this function has returned.
%
In case \fcn{getaddrinfo()} returns a nonzero (error) value,
passing it to
\fcnsys{gai\_strerror()} yields a string that describes what went wrong.

Generally speaking, \fcn{getaddrinfo()}
takes the name of a host/service pair as input,
and returns a linked list of structures containing everything needed
to create a socket to connect to the named host/service, including:
address/protocol family (v4 or v6), socket type
(e.g. stream or datagram), protocol (TCP or UDP for the Internet
protocol family), and numeric socket address.
%
Each entry in the linked list is placed
into an \type{addrinfo} structure, declared as follows:
%
\begin{inlinecode}
struct addrinfo {
  int ai_flags;             // Flags to control info resolution
  int ai_family;            // Family:  AF_INET, AF_INET6, AF_UNSPEC
  int ai_socktype;          // Socket type:  SOCK_STREAM, SOCK_DGRAM
  int ai_protocol;          // Protocol: 0 (default) or IPPROTO_XXX
  socklen_t ai_addrlen;     // Length of socket address ai_addr
  struct sockaddr *ai_addr; // Socket address for socket
  char *ai_canonname;       // Canonical name
  struct addrinfo *ai_next; // Next addrinfo in linked list
};
\end{inlinecode}
%
The \type{ai\_sockaddr} field contains a \type{sockaddr} of the
appropriate type, with (numeric) address and port information filled in.
It should be obvious which fields contain
the address family, socket type, and protocol information. (The flags
field is not used in the result; we will discuss its use shortly.)
Actually, the results are returned in a pointer to a linked list of
\type{addrinfo}
structures; the \varsys{ai\_next} field contains the pointers for this list.

Why a linked list?  Two reasons.  First, for each combination of
host and service, there might be several different combinations of
address family (v4 or v6) and socket type/protocol (stream/TCP or
datagramUDP) that represent possible endpoints.  For example, the host
``server.example.net'' might have instances of the ``spam'' service
listening on port 1001 on both IPv4/TCP and IPv6/UDP.  The \fcn{getaddrinfo()}
function returns both of these.  The second reason is that
a hostname can map to multiple IP addresses; \fcn{getaddrinfo()}
helpfully returns all of these.

Thus, \fcn{getaddrinfo()} returns all the viable combinations for a given
hostname, service pair.
But wait---what if you don't \emph{need} options,
and you know exactly what you want in advance?
You don't want to have to write code that searches through the
returned list for a particular combination---say, IPv4/TCP.
That's where the third parameter of
\fcn{getaddrinfo()} comes in!  It allows you to tell the system to
\emph{filter the results for you}.  We'll see how it is used in
our example program, \file{GetAddrInfo.c}.

\jcode{GetAddrInfo.c}{code/GetAddrInfo.c}{1}{1}

\begin{topcode}

\tlcitems{Application setup and parameter parsing}{8--12}

\tlcitems{Construct address specification}{14--19}

The \var{addrCriteria} structure will indicate
what kinds of results we are interested in.

\begin{bottomcode}

\blcitems{Declare and initialize \typesys{addrinfo} structure}{15--16}

\blcitem{Set address family}{17}

We set the family to \constsys{af\_unspec}, which allows the returned
address to come from any family (in particular, either
\const{af\_inet} or \const{af\_inet6}).

\blcitem{Set socket type}{18}

We want a stream/TCP endpoint, so we set this to \const{sock\_stream}.
The system will filter out results that use different protocols.

\blcitem{Set protocol}{19}

We want a TCP socket, so we set this to \constsys{ipprot\_tcp}.
Since TCP is the default protocol for stream sockets, leaving this
field zero would have the same result.

\end{bottomcode}

\tlcitems{Fetch address information}{21--26}

\begin{bottomcode}

\blcitem{Declare pointer for head of result linked list}{22}

\blcitem{Call \fcnrefsys{getaddrinfo()}}{24}

We pass the desired host name, port, and the constraints encoded in
the \var{addrCriteria} structure.

\blcitems{Check return value}{25--26}

\fcnrefsys{getaddrinfo()} returns 0 if successful.  Otherwise, the
return value indicates the specific error.
The auxiliary function \fcnrefsys{gai\_strerror()} returns a character string
error message explaining the given error return value.  Note that
these messages are different from the normal \varsys{errno}-based messages.

\end{bottomcode}

\tlcitems{Print addresses}{28--32}

Iterate over the linked list of addresses, printing each to the console.  
The function
\fcnref{PrintSocketAddress()} takes an address to print and the stream
on which to print. We present its code, which is in
\file{AddressUtility.c}, later in this chapter.

\tlcitem{Free address linked list}{34}

The system allocated storage for the linked list of
\typesys{addrinfo} structures it returned (if any).
We must call the auxiliary function \fcnrefsys{freeaddrinfo()} to free that
memory when we are finished with it.

\end{topcode}

The program \file{GetAddrInfo.c} takes two command-line parameters, a
host name (or address) and a service name (or port number), and prints
the address information returned by \fcn{getaddrinfo()}.
Suppose you want to find an address for the service named ``whois'' on
the host named ``localhost'' (i.e., the one you are running on).
Here's how you use it:
%
\begin{shell}
\prompt \typed{GetAddrInfo localhost whois} \\
\response{127.0.0.1-43}
\end{shell}
To find the service ``whois'' on the host
``pine.netlab.uky.edu'', do this:

\begin{shell}
\prompt \typed{GetAddrInfo pine.netlab.uy.edu whois} \\
\response{128.163.140.219-43}
\end{shell}
%
The program can deal with any combination of name and numerical
arguments:
\begin{shell}
\prompt \typed{GetAddrInfo 169.1.1.100 time}\\
\response{169.1.1.100-37}\\
\prompt \typed{GetAddrInfo
  FE80:0000:0000:0000:0000:ABCD:0001:0002:0003 12345}\\
\response{fe80::abcd:1:2:3-12345}
\end{shell}
These examples all return a single answer.  But as we noted above,
some names have multiple numeric addresses associated with them.
For example, ``google.com'' is typically associated with a number of
Internet addresses.  This allows a service (e.g., search
engine) to be placed on multiple hosts.
Why do this?  One reason is robustness.
If any single host fails, the service continues because the client can
use any of the hosts providing the service.  Another advantage is
scalability.  If clients randomly select the numeric address (and
corresponding host) to use, we can spread the load over multiple
servers.  The good news is that \fcnrefsys{getaddrinfo()}
returns \emph{all\/} of the addresses
to which a name maps.  You can experiment with this by
executing the program with the names of popular web sites.
(Note that using 0 for the second argument causes only the address
information to be printed.)

\subsection{Details, Details}
\label{sect:gaiadv}
As we noted above, \fcn{getaddrinfo()} is a something of a ``Swiss
Army Knife'' function.  We'll cover some of the subtleties of
its capabilities here.
%
Beginning readers may wish to skip this section and come back
to it later.

The third argument (\type{addrinfo} structure) tells the system what
kinds of endpoints the caller is interested in.  In
\file{GetAddrInfo.c}, we set this parameter to indicate
that any address family was acceptable,
and we wanted a stream/TCP socket.  We could've specified
datagram/UDP instead, or specified a particular address family (say,
\const{af\_inet6}), or we could have set the \varsys{ai\_socktype} and
\var{ai\_protocol} fields to zero, indicating that we wanted to
receive \emph{all\/} possibilities.  It is even possible to pass a
NULL pointer for the third argument; the system is supposed to treat
this case as if an \type{addrinfo} structure had been passed with
\var{ai\_family} set to \constsys{af\_unspec} and everything else set
to 0.


The \var{ai\_flags} field in the third parameter provides
additional control over the behavior of \fcnrefsys{getaddrinfo()}.
It is an integer, individual bits of which are interpreted as boolean
variables by the system.  The meaning of each flag is given below;
flags can be combined 
using the bitwise OR operator ``$\mid$'' (see Section~\ref{sect:bitdiddling} for
how to do this).

\begin{description}
% XXXX I changed all these to capitals, because small caps was not
% working in the arguments to \item.
\item[AI\_PASSIVE]
If \param{hostStr} is NULL when this flag is set, any returned 
\typesys{addrinfo}s will
have their addresses set to the appropriate ``any'' address
constant---\constsys{inaddr\_any} (IPv4) 
or \constsys{in6addr\_any\_init} (IPv6).

\item[AI\_CANONNAME]
Just as one name can
resolve to many numeric addresses, multiple names can resolve to the
same IP address.
%
However, one name is usually defined to be the official
(``canonical'') name.  By setting this flag in \varsys{ai\_flags}, 
we instruct \fcn{getaddrinfo()} to
return a pointer to the canonical name (if it exists) in
the \typesys{ai\_canonname} field of the first \typesys{struct addrinfo} of
the address linked list.

\item[AI\_NUMERICHOST] 
This flag causes an error to be returned if
\param{hostStr} does not point to a string in valid numeric address
format.  Without this flag,
if the \param{hostStr} parameter points to something that is not a
valid string representation of a numeric address,
an attempt will be made to resolve it via the name system; this can
waste time and bandwidth on useless queries to the name service.
If this flag is set, a given valid address string is simply converted and
returned, a la \fcnsys{inet\_pton()}.

\item[AI\_ADDRCONFIG]
If set, \fcn{getaddrinfo()} returns addresses of a particular
family only if the system has an interface configured for that
family.  So an IPv4 address would be returned only if the system has
an interface with an IPv4 address, and similarly for IPv6.

\item[AI\_V4MAPPED]
If the \varsys{ai\_family} field contains \const{af\_inet6}, and
no matching IPv6 addresses are found,
then \fcnrefsys{getaddrinfo()} returns IPv4-mapped IPv6 addresses.
This is a technique that can be used to provide limited interoperation
between IPv4-only and IPv6 hosts.
% XXX elaborate
\end{description}

\section{Writing Address-Generic Code}
\label{sect:address-independence}

A famous bard once wrote ``To v6 or not to v6, that is the question.''
Fortunately, the Socket interface allows us to postpone answering that
question until execution time.  In our earlier TCP client and server
examples, we specified a particular IP protocol version to both the socket
creation and address string conversion functions using
\constsys{af\_inet} or \constsys{af\_inet6}.  However,
\fcnsys{getaddrinfo()} allows us to write code
that works with either address family, without having to
duplicate steps for each version.  In this section we'll use
its capabilities to modify our version-specific client and server
code to make them generic.

Before we do that, here's
a handy little method that prints a socket address
(of either flavor).
Given a \type{sockaddr} structure containing an IPv4 or IPv6 address,
it prints the address to the given output stream, using the proper
format for its address family.
Given any other kind of address, it prints an error string.
This function takes a generic \typesys{struct sockaddr}
pointer and prints the address to the specified stream.  You can find
our implementation of \fcnref{PrintSocketAddr()} in \file{AddressUtility.c}
with the function prototype included in \file{Practical.h}.

\jcode[38]{PrintSocketAddr()}{code/AddressUtility.c}{1}{6}

\subsection{Generic TCP Client}

Using \fcnrefsys{getaddrinfo()}, we can write clients
and servers that are not specific to one IP version or the other.
Let's begin by converting our TCP client to make it
version-independent; we'll drop the version number and
call it \file{TCPEchoClient.c}.
The general strategy is to set up arguments to \fcn{getaddrinfo()}
that make it return both IPv4 and IPv6 addresses and use the
first address that works.
% XXXX persistent client
Since our address search functionality may be useful elsewhere, we factor
out the code responsible for creating
and connecting the client socket, placing it in a separate function,
\fcnref{SetupTCPClientSocket()}, in TCPClientUtility.c.  The setup function
takes a host and service, specified in a string, and returns a connected
socket (or -1 on failure).  The host or service may be specified as NULL.

\jcode{TCPClientUtility.c}{code/TCPClientUtility.c}{1}{1}

\begin{topcode}

\tlcitems{Resolve the host and service}{10--20}

The criteria we pass to \fcn{getaddrinfo()}
specifies that we don't care which protocol is used
(\constsys{af\_unspec}), but the socket address is for TCP 
(\constsys{sock\_stream}/\constsys{ipproto\_tcp}).  

\tlcitems{Attempt to create and connect a socket from the list of addresses}{22-40}

\begin{bottomcode}

\blcitems{Create appropriate socket type}{25--27}

\fcnrefsys{getaddrinfo()} returns the matching domain (\constsys{af\_inet}
or \constsys{af\_inet6}) and socket type/protocol.  We pass this
information on to \fcnrefsys{socket()} when creating the new socket.
If the system cannot create a socket of the specified type, we move on
to the next address.

\blcitems{Connect to specified server}{30--34}

We use the address obtained from \fcnrefsys{getaddrinfo()} to attempt to 
connect to the server.  If the connection succeeds, we exit the address
search loop.  If the connection fails, we close the socket and
try the next address.

\end{bottomcode}

\tlcitem{Free address list}{37}

To avoid a memory leak, we need to free the address linked list created by
\fcnrefsys{getaddrinfo()}.

\tlcitem{Return resulting socket descriptor}{38}

If we succeed in creating and connecting a socket, return the socket descriptor.
If no addresses succeeded, return -1.

\end{topcode}

\jcode{TCPEchoClient.c}{code/TCPEchoClient.c}{1}{1}

After socket creation, the remainder of the program is identical to the 
version-specific clients.
There is one caveat that must be mentioned with respect to this code.
In line~25 of \fcnref{SetupTCPClientSocket()}, we pass the 
\varsys{ai\_family} field of the returned \type{addrinfo} structure 
as the first  argument to \fcnrefsys{socket()}.
Strictly speaking, this value identifies an
\emph{address family} (\constsys{af\_xxx},
whereas the first argument of \fcn{socket}
indicates the desired \emph{protocol family\/} of the socket
(\constsys{pf\_xxx}).
In all implementations with which we have experience, these two
families are interchangeable---in particular \constsys{af\_inet}
and \constsys{pf\_inet} are defined to have the same value, as are
\constsys{pf\_inet6} and \constsys{af\_inet6}.
%
% XXX why doesn't this work??
\callout{Our generic code depends on this fact.}
%
The authors contend that these definitions will not change, but
\label{page:AFPFassumpt}
feel that full disclosure of this assumption (which allows more
concise code) is important.
Elimination of this assumption is straightforward enough to be left as
an exercise.

\subsection{Generic TCP Server}

Our protocol-independent TCP echo server uses similar adaptations to
those in the client.
%
Recall that the typical server binds to \emph{any} available local
address.  To accomplish this, we (1) specify the \constsys{ai\_passive}
flag and (2) specify \constsys{null} for the hostname.  Effectively,
this gets an address suitable for passing to \fcnrefsys{bind()},
including a wildcard for the local IP
address---\constsys{inaddr\_any} for IPv4 or
\constsys{in6addr\_any} for IPv6).  For systems that support
\emph{both\/} IPv4 and IPv6, IPv6 will generally be returned
first by \fcn{getaddrinfo()}, because it offers more options for
interoperability.  Note, however, that the problem of
which options should be selected to maximize connectivity depends on
the particulars of the environment in which the server operates---from
its name service to its Internet Service Provider.
\callout{The approach we present here is essentially
the simplest possible, and is likely not adequate for production servers
that need to operate across a wide variety of platforms.}
See the next section for additional information.

Like in our protocol-independent client, we've factored out establishing a socket
into a separate function, \fcnref{SetupTCPServerSocket()}, in \file{TCPServerUtility.c}.
This setup function iterates over the addresses returned from \fcnref{getaddrinfo()},
stopping when it can successfully bind and listen or when it's out of addresses.

\jcode[51]{SetupTCPServerSocket()}{code/TCPServerUtility.c}{1}{8}

\noindent We also factor out handling accepting client connections into a separate
function, \fcnref{AcceptTCPConnection()}, in \file{TCPServerUtility.c}.

\jcode[54]{AcceptTCPConnection()}{code/TCPServerUtility.c}{1}{8}

\noindent Note that we use \fcnrefsys{getsockname()} to print the
local socket address.  When you execute \file{TCPEchoServer.c}, it
will print the wildcard local network address.  Finally, we use our new 
functions in our protocol-independent echo server.

\jcode{TCPEchoServer.c}{code/TCPEchoServer.c}{1}{1}

\subsection{IPv4-IPv6 Interoperation}
%
Our generic client and server are oblivious to whether they are
using IPv4 or IPv6 sockets.
%
An obvious question is ``What if one is
using IPv4 and the other IPv6?''  The answer is that if
(and only if) the program
using IPv6 is a \emph{dual-stack\/} system---that is, supports
\emph{both\/} verson 4 and version 6---they should be able to interoperate.
%
The existence of the special ``v4-to-v6-mapped'' address class makes
this possible.  This mechanism allows an IPv6 socket to be connected to
an IPv4 socket.  A full discussion of how the implications of this and
how it works is beyond the scope of this book, but the basic idea is
that the IPv6 implementation in a dual-stack system recognizes that
communication is desired between an IPv4 address and an IPv6 socket,
and translates the IPv4 address into a ``v4-to-v6-mapped'' address.
Thus, each socket deals with an address in its own format.

For example, if the client is a v4 socket and the server is listening
on a v6 socket in a dual-stack platform, when the connection request
comes in the server-side implementation
will automatically do the conversion and tell the server
that it is connected to a v6 socket with a v4-mapped address.
(Note that there is a bit more to it than this; in particular the
server side implementation will first try to match to a socket bound
to a v4 address, and only do the conversion if it fails to find a
match; see Chapter~\ref{chap:under} for more details.)

If the server is listening on a v4 socket, the client is trying to
connect from a v6 socket on a dual-stack platform, \emph{and\/} the
client has not bound the socket to a particular address before calling
\fcnrefsys{connet()}, the client-side
implementation will recognize that it is connecting to an IPv4 address
and assign a v4-mapped IPv6 address to the socket at \fcn{connect()} time.
The stack will ``magically'' convert the assigned address to an IPv4
address when the connection request is sent out.
%
Note  that, in both cases, the message that goes over the network is
actually an IPv4 message.

While the v4-mapped addresses provide a good measure of
interoperability, the reality is that the space of possible scenarios
is very large when one considers v4-only hosts, v6-only hosts, hosts
that support IPv6 but have no configured IPv6 addresses, and hosts
that support IPv6 and use it on the local network, but have no
wide-area  IPv6 transport available (i.e., their providers do not
support IPv6).   Although our example code---a client that
tries all possibilities returned by \fcn{getaddrinfo()}, and a server
that sets \constsys{ai\_passive} and
binds to the first address returned by \fcn{getaddrinf()}---covers the
most likely possibilities, \callout{production code
needs to be very carefully designed to maximize the
likelihood that clients and servers will find each other under all
conditions}.  The details of achieving this are beyond the scope of
this book; the reader should 
refer to RFC~4038~\cite{RFC4038} for more details.

\section{Getting Names from Numbers}

So we can get an Internet address from a host name, but can we perform
the mapping in the other direction (host name from an Internet
address)?
The answer is ``usually''.
There is an ``inverse'' function called \fcnrefsys{getnameinfo()}, which
takes a \typesys{sockaddr} address structure (really a \typesys{struct
sockaddr\_in} for IPv4 and \typesys{struct sockaddr\_in6} for IPv6)
and the address length. 
The function returns a corresponding node and service name in the form
of a null-terminated character string---if the mapping between the
number and name is stored in the name system.
%
Callers 
of \fcnrefsys{getnameinfo()} must preallocate the space for the node
and service names and pass in a pointer and length for the space.  The maximum
length of a node 
or service name is given by the constants \constsys{NI\_MAXHOST} and
\constsys{NI\_MAXSERV},  
respectively.  If the caller specifies a length of 
0 for node and/or service, \fcnrefsys{getnameinfo()} does not return
the corresponding value. 
This function returns 0 if successful and a non-zero value if failure.
The non-zero failure return code can again be passed to
\fcnrefsys{gai\_strerror()} to 
get the corresponding error text string.

\begin{inlinefcn}
\type{int} \fcnrefsys{getnameinfo}(
\type{const struct sockaddr *}\param{address},
\type{socklen\_t} \param{addressLength},\\
\hspace*{1.1in}\type{char *}\param{node}, \type{socklen\_t} \param{nodeLength},
\type{char *} \param{service},\\
\hspace*{1.1in}\type{socklen\_t}
\param{serviceLength},\type{int} \param{flags})
\end{inlinefcn}
%
As with \fcn{getaddrinfo()},
several flags control the behavior of 
\fcnrefsys{getnameinfo()}.  They are described below;
% XXXX OROP
as before, some of them can be combined using bit-wise OR (``$\mid$'').

\begin{description}
\item[NI\_NOFQDN]
Return only the host name, not FQDN (``Fully Qualified Domain Name''), for
local hosts.  (The FQDN contains all parts, for example
``protocols.cs.uky.edu'', while the host name is only the first
part, for example ``protocols''.)
\item[NI\_NUMERICHOST]
Return the numeric form of the address instead of the name.  This
avoids potentially expensive name service lookups if you just want to use
this service as a substitute for \fcnrefsys{inet\_ntop()}.
\item[NI\_NUMERICSERV]
Return the numeric form of the service instead of the name.
\item[NI\_NAMEREQD]
Return an error if a name cannot be found for the given address.
Without this option, the numeric form of the address is returned.
\item[NI\_DGRAM]
Specifies datagram service; the default behavior assumes a stream
service. In some cases, a service has different
port numbers for TCP and UDP.
\end{description}

What if your program needs its own host's name?
\fcnrefsys{gethostname()} takes a buffer and buffer length
and copies the name of the host on which the calling program is running into
the given buffer.

\begin{inlinefcn}
\type{int} \fcnrefsys{gethostname}(\type{char *}\param{nameBuffer}, 
\type{size\_t} \param{bufferLength})
\end{inlinefcn}

\begin{exercises}
\item
\file{GetAddrInfo.c} requires two arguments.  How could you get it to
resolve a service name if you don't know any host name?
\item
Modify \file{GetAddrInfo.c} to take an optional third argument,
containing ``flags'' that are passed to \fcn{getaddrinfo()} in the
\varsys{ai\_flags} field of the \varsys{addrinfo} argument.
For example, passing ``-n'' as the third argument should result in the
\varsys{ai\_numerichost} flag to be set.
\item
Does \fcn{getnameinfo()} work for IPv6 addresses as well as IPv4?
What does it return when given the address ::1?
\item
Modify the generic \file{TCPEchoClient} and \file{TCPEchoServer} to
eliminate the assumption mentioned on page~\pageref{page:AFPFassumpt}.
\end{exercises}

% XXX DEAD TEXT
%% So if most network application users expect to use a name, how do we
%% resolve a host name to a numeric address or a service name to a port number,
%% as expected by the Sockets interface?  Well, it turns out that
%% \fcnrefsys{getaddrinfo()} is something of a ``Swiss army knife'' of
%% resolution functions: it can be used to resolve both host names like
%% www.mkp.com, and service names, like ``echo''.
%% So what must we do to make
%% \file{GetAddrInfo.c}, \file{TCPEchoClient.c}, and
%% \file{TCPEchoServer.c} take names?  Nothing!  Because we used
%% \fcnrefsys{getaddrinfo()}, these applications already take either
%% names or addresses.  executing these applications using names in place
%% of numeric addresses.

 
%% Perhaps because there is less need for the additional level of
%% indirection, there is no global database that maps names to port
%% numbers.  In UNIX, the information returned about the mapping between services
%% and port numbers usually comes from a local database or the
%% \file{/etc/services} file.  It is important to realize that
%% this information
%% may not be valid for a host that is managed by a different
%% administration; however, it is arguably better than nothing.

