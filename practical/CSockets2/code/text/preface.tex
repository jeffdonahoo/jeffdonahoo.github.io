\newpage
\thispagestyle{empty}
%\vspace{1in}
\vfill
\begin{center}
{\Huge\textbf{TCP/IP Sockets in C}} \\
{\Large\textbf{Practical Guide for Programmers, Second Edition}  \\[.2in]}
{\bf Michael J. Donahoo \\
Kenneth L. Calvert \\[.2in]}
\end{center}

\section*{Preface to the Second Edition}

When we wrote the first edition of this book, it was not very common
for college courses on networking to include programming components.
That seems difficult to believe now, when the Internet has become so
important to our world, and the pedagogical benefits of hands-on programming
and real-world protocol examples are so widely accepted.
%
Although there are now other languages that provide access
to the Internet, interest in the
original C-based \emph{Berkeley Sockets} remains high.
The Sockets  API (application programming interface) for
networking was developed at UC Berkeley in the 1980's
for the BSD flavor of Unix---one of the very first
examples of what would now be called an ``open source'' project.

The Sockets API and the Internet both grew up in a world of many
competing protocol families---IPX, Appletalk, DECNet, OSI, and SNA in
addition to TCP/IP---and Sockets was designed to support them all.
Fewer protocol families were in common use by the time we wrote the
first edition of this book, and the number today is even smaller.
Nevertheless,  as we predicted in the first edition, the sockets API remains
important for those who want to design and build distributed
applications that use the Internet---i.e., which use TCP/IP.
And the interface has proven robust enough
to support the new version of the Internet Protocol (IPv6), which is
now supported on virtually all common computing platforms.

Two main considerations motivated this second edition.
First, based on our own experience and feedback from others,
we found that some topics needed to be presented in more
depth and that others needed to be expanded.
%
The second consideration is the increasing acceptance and use of IP
version~6, which is now supported by essentially all current end
system platforms.  At this writing, it is not possible to use IPv6 to
exchange messages with a large fraction of hosts on the Internet, but
it \emph{is\/} possible to assign an IPv6 address to many of them.
While it is still too early to tell whether IPv6 will take over the
world, it is not too early to start writing applications to be
prepared.

\subsection*{Changes from the First Edition}

We have updated and considerably expanded most of the material, having
added two chapters.  Major changes from the first edition include:
\begin{itemize}
\item
IP version 6 coverage.  We now include three kinds of code:
IPv4-specific, IPv6-specific, and generic.  The code in the
later chapters is designed to work with either protocol version on
dual-stack machines.
\item
An additional chapter on socket programming in C++
(contributed by David B.\ Sturgill).
% XXXXXXXX CHECK IT
The PracticalSocket library provides wrappers for basic socket functionality.
These allow an instructor to teach socket programming to
students without C programming background by giving them a library and
then gradually peeling back the layers.  Students can start
developing immediately after understanding addresses/ports and
client/server.  Later they can be shown the details of socket programming by
peeking inside the wrapper code.
Those teaching a subject that uses networking (e.g., OS) can
use the library and only selectively peel back the cover.
\item
Enhanced coverage of data representation issues and strategies for
organizing code that sends and receives messages.  In our
instructional experience we find that students have less and less
understanding of how data is actually stored in memory,\footnote{We
speculate that this is due to the widespread use of C++ and Java,
which hide such details from the programmer, 
in undergraduate curricula.} so we have
attempted to compensate with more discussion of this important
issue.  At the same time, internationalization will only increase in
importance, and thus we have included basic coverage of
wide characters and encodings.
\item
Omission of the reference section.  The descriptions of
most of the functions that make up the Sockets API have been collected
into two chapters.  However, with so many on-line sources of reference
information---including ``man pages''---available,
we chose to leave out the complete listing of the API in favor of
more code illustrations.
\item
Highlighting important but subtle facts and caveats.  Typographical devices
call out important concepts and information that might otherwise
be missed on first reading.
\end{itemize}

Although the scope of the book has expanded, we have not included
everything that we might have (or even that we were asked to include);
examples of topics left for more comprehensive texts (or the
next edition): raw sockets and programming with Winsock.

%% Excellent reference books on TCP/IP socket programming
%% exist, but they are too large and comprehensive to be considered as a
%% supplement to a networking text.  UNIX ``man pages'' are great for
%% reference but do not make a very good tutorial.

%% Enabling students to get their hands on real network services via the
%% sockets interface has several benefits.  First, for a surprising
%% number of people, socket programming is the first exposure to concrete
%% realizations of concepts previously seen only in the abstract.
%% Dealing with the very real consequences of messy details, such as the
%% layout of data structures in memory, seems to trigger a kind of
%% epiphany in some students, and this experience has consequences far
%% beyond the networking course.  Second, we find that students who
%% understand how application programs \emph{use} the services of TCP/IP
%% generally have an easier time grasping the principles of the
%% underlying protocols that \emph{implement} those services. Finally,
%% basic socket programming skills are a springboard to more advanced
%% assignments, which support learning about routing algorithms,
%% multimedia protocols, medium access control, and so on.

\subsection*{Intended Audience}

We originally wrote this book so we'd have something to hand our students
when we wanted them to learn socket programming, so we wouldn't have
to take up valuable class time teaching it.  In the years since
the first edition, we have learned a good deal
about the topics that students need lots of help on, and those where
they do not need as much handholding.  We also found that
our book was appreciated at least as much by practitioners who were
looking for a gentle introduction to the subject.
%
Therefore this book is aimed simultaneously at two general audiences:
students in introductory courses in computer networks (graduate or
undergraduate) with a programming component, and
practitioners who want write their own programs that
communicate over the Internet.
For students, it is intended as a supplement, not as a primary text
about networks.
Although this second edition is significantly bigger in size and scope than
the first, we hope the book will still be considered a good value in
that role.
%
For practitioners who just want to write some useful code, it should
serve as a standalone \emph{introduction}---but readers in that
category should be warned that this book will not make them
experts. 
%
Our philosophy of learning by 
doing has not changed, nor has our approach of providing a concise
tutorial sufficient to get one started learning on one's own, and
leaving the comprehensive details to other authors.
For both audiences our goal is to
take you far enough so that you can start experimenting and
learning on your own.

\subsection*{Assumed Background}

We assume basic programming skills and experience with C and UNIX.
You are expected to be conversant with C concepts such as pointers
and type casting, and should have a basic understanding of
the binary representation of data.  Some of our examples are factored
into files that should be compiled separately; we assume that you can deal with
that.

Here is a little test: If you can puzzle out what the following code
fragment does, you should have no problem with the code in this book:

\begin{inlinecode}
  typedef struct {
    int a;
    short s[2];
  } MSG;

  MSG *mp, m = {4, 1, 0};
  char  *fp, *tp;
  mp = (MSG *) malloc(sizeof(MSG));
  for (fp = (char *)m.s, tp = (char *)mp->s; tp < (char *)(mp+1);)
     *tp++ = *fp++;
\end{inlinecode}

If you do not understand this fragment, do not despair (there is
nothing quite so convoluted in our code), but you might want to refer
to your favorite C programming book to find out what is going on here.

You should also be familiar with the UNIX notions of process/address
space, command-line arguments, program termination, and regular file
input and output. The material in Chapters 4 and 5 assumes a somewhat
more advanced grasp of UNIX.  Some prior exposure to networking concepts
such as protocols, addresses, clients, and servers will be helpful.

\subsection*{Platform Requirements and Portability}

Our presentation is UNIX-based.  When we were developing this
book, several people urged us to include code for Windows as well as
UNIX.  It was not possible to do so for various reasons including the
target length (and price) we set for the book.

For those who only have access to Windows platforms, please note that
the examples in the early chapters require minimal modifications to work
with WinSock.  (You have to change the include files and add a setup
call at the beginning of the program and a cleanup call at the end.)
Most of the other examples also require very slight additional
modifications.  However, some are so dependent on the UNIX
programming model that it does not make sense to port them to WinSock.
%
WinSock-ready versions of the other examples, as well as detailed
descriptions of the code modifications required, are available from
the book's Website at \emph{www.mkp.com/socket} [XXX URL].  Note also that
almost all of our example code works with minimal modifications under
the \textbf{Cygwin} Unix library package for Windows, which is
available online.

For this second edition, we have adopted the C99 language standard.
This version of the language is supported by most compilers and offers
so many readability-improving advantages---including line-delimited
comments, fixed-size integer types,
and declarations anywhere in a block---that we could not
justify not using it.

Our code makes use of the ``Basic Socket Interface Extensions for
IPv6''~\cite{RFC3493}.
Among these extensions is a new and different interface to the name system.
Because we rely completely on this new interface
(\fcn{getaddrinfo()}), our generic code may not run on some older
platforms.  However, we expect that most modern systems will run our
code just fine.

The example programs included here have all been tested (and
should compile and run without modification) on both Linux and
MacOS.
% XXXXXXX
Header (\texttt{.h}) file locations and dependencies are, alas, not quite
standard and may require some fiddling on your system. Socket option
support also varies widely across systems; we have tried to focus on
those that are most universally supported.  Consult your API
documentation for system specifics.  (By ``API documentation'' we mean
the ``man pages'' for your system.  To learn about this, type ``man man''.)

Please be aware that although we strive for a basic level of
robustness, the primary goal of our
code examples is pedagogy, and the code is \textbf{not production
quality.}  We have sacrificed some robustness for brevity and clarity,
especially in the generic server code.  (It turns out to be rather
nontrivial to write a server that works under all combinations
of IPv4 and IPv6 protocol configurations and also maximizes the
likelihood of successful client connection under all circumstances.)

\subsection*{This Book Will Not Make You an Expert!}

We hope this
second edition will be useful as a resource, even to those who already
know quite a bit about sockets.
As with the first edition, we learned some things in writing it.
But becoming an expert
takes years of experience, as well as other, more comprehensive
sources~[\cite{ComerV3}, \cite{StevensUNP}]. 

%% The first part of the book is a tutorial, which begins with ``just
%% enough'' of the big picture, then quickly gets into code basics via
%% some example programs.  The first four chapters aim to get you quickly
%% to the point of constructing simple clients and servers, such as might
%% be needed to complete introductory assignments.  After that we branch
%% out in Chapter~5, introducing some of the many different ways to use
%% sockets.  Chapter~6 returns to basic socket operation to provide more
%% in-depth coverage of some of the underlying mechanisms and some
%% pitfalls to watch out for.  Chapter~7 describes domain names and how
%% they can be used to obtain Internet addresses.

The first chapter is intended to give ``just enough'' of the big
picture to get you ready to write code.  Chapter~\ref{chap:tcp}
shows you how to write TCP clients and servers using either IPv4 or IPv6.
Chapter~\ref{chap:addrindep} shows how to make your clients and
servers use the network's name service, and also shows how to make
them IP-version-independent.  Chapter~\ref{chap:udp} covers UDP.
Chapters~\ref{chap:encoding} and~\ref{chap:advanced} provide
background needed to write more programs, while
Chapter~\ref{chap:under} gives relates some of what is going on in the
Sockets implementation to the API calls; these three are essentially
independent and may be presented in any order.
Finally, Chapter~\ref{chap:cpp} presents a C++ class library that
provides simplified access to socket functionality.

Throughout the book, certain statements are highlighted like this:
\callout{This book will not make you an expert!}  Our goal is to bring
to your attention those subtle but important facts and ideas that one
might miss on first reading.  The marks tell you to ``\textbf{note
well} whatever is in bold.


\subsection*{Acknowledgments}

Many people contributed to making this
book a reality.  In addition to all those who helped us with the first
edition (Michel Barbeau, Steve Bernier,
Arian Durresi,
Gary Harkin, Ted Herman,
Lee Hollaar, David Hutchison, Shunge Li, Paul Linton,
Ivan Marsic, Willis Marti,
Kihong Park, Dan Schmitt, Michael Scott,  Robert Strader, 
Ben Wah,
and Ellen Zegura), we especially thank David B.\ Sturgill, who
contributed code and text for Chapter~\ref{chap:cpp}, 
%
Bobby Krupczak for
his help reviewing the draft of this second edition.
Finally, to the folks at Morgan Kaufmann/Elsevier---Rick Adams, our
editor, Maria Alonso, and our production editor [N.N.]---thank you for
your patience, help, and caring about the quality of our book.

\subsection*{Feedback}

We are very interested in weeding out
errors and otherwise
improving future editions/printings, so if you find one, please send
email to either of us.  We will maintain an errata list on the book's
Web page.

\vspace{1.4ex}
\noindent
\textsl{
M.J.D. \texttt{jeff\_donahoo@baylor.edu}\\
K.L.C \texttt{calvert@netlab.uky.edu}
}

%\copyrightnotice

