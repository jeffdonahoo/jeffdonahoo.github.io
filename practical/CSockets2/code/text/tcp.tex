% OUTLINE
% 1.  IPv4 TCP Client (TCPEchoClient4.c)
% 2.  IPv4 TCP Server (TCPEchoServer4.c)
% 3.  Creating and Destroying Sockets
% 	3.1.  socket()
% 	3.2.  close()
% 4.  Specifying Addresses
% 	4.1.  Address Structures
% 		- sockaddr
% 		- sockaddr_in
% 		- sockaddr_in6
% 		- sockaddr_storage
% 	4.2. Binary/String Address Conversion
% 		- inet_pton()
% 		- inet_ntop()
% 	4.3. Binary/String Port Conversion
%   4.4. Getting a Socket Local and Foreign Address
% 5.  Preparing for Communication
%   5.1  connect()
%   5.2.  bind()
% 6.  Accepting Incoming Connections
% 	- listen()
% 	- accept()
% 7.  Communication
% 	- send()
% 	- recv()
% 	- Comments on streams
% 8.  Using IPv6 (TCPEchoServer6.c)
\chapter{Basic TCP Sockets}
\label{chap:tcp}

It's time to learn about writing your own socket applications.
We'll start with TCP.
By now you're probably ready to get your hands dirty with some
actual code so we begin by going through
a working example of a TCP client and server.
Then we present the details of the socket API used in basic TCP.
To keep things simpler, we'll present code initially that works for
one  particular version of IP: IPv4, which at the time this is written is
still the dominant version of the Internet Protocol, by a wide margin.
At the end of this chapter we present the (minor)
modifications required to write IPv6 versions of our
clients and servers.  In Chapter~\ref{chap:addrindep}
we will demonstrate the creation of protocol-independent applications.
\marginpar{Testing!}

Our example client and server implement the \defn{echo} protocol.  It
works like this: the client connects to the server and sends its data.  The
server simply echoes whatever it receives back to the client and
disconnects.
%
In our application, the data that the client sends is a string provided as a
command-line argument.  Our client will print the data
it receives from the server so we can see 
what comes back.  Many systems include an echo service for debugging
and testing purposes.

\section{IPv4 TCP Client}

The distinction between client and server is important
because each uses the sockets interface differently at certain steps
in the communication.  We first focus on the client.  Its job is to
initiate communication with a server that is passively waiting to be
contacted.  

The typical TCP client's communication involves four basic steps:
%
\begin{enumerate}

\item Create a TCP socket using \fcnrefsys{socket()}.

\item Establish a connection to the server using
\fcnrefsys{connect()}.

\item Communicate using \fcnrefsys{send()} and
\fcnrefsys{recv()}.

\item Close the connection with \fcnrefsys{close()}.
%
\end{enumerate}

\noindent \file{TCPEchoClient4.c} is an implementation of a TCP echo client for
IPv4.

\jcode{TCPEchoClient4.c}{code/TCPEchoClient4.c}{1}{1}


Our \file{TCPEchoClient4.c} does the following:

\begin{topcode}

\tlcitems{Application setup and parameter parsing}{1--20}

\begin{bottomcode}

\blcitems{Include files}{0--7}

These header files declare the standard functions and constants of the
API.  Consult your documentation (e.g., man pages) for the appropriate
include files for socket functions and data structures on your system.  We
utilize our own include file, \file{Practical.h}, with prototypes for our own
functions, which we describe below.

\blcitems{Typical parameter parsing and sanity checking}{13--20}

The IPv4 address and string to echo are passed in as the first two parameters. 
Optionally, the client takes the server port as the third parameter.  If no
port is provided, the client uses the well-known echo protocol port, 7.

\end{bottomcode}

\tlcitems{TCP socket creation}{22--25}

We create a socket using the \fcnrefsys{socket} function.  The socket is for
IPv4 (\constsys{af\_inet}) using the stream-based protocol
(\constsys{sock\_stream}) called TCP (\constsys{ipproto\_tcp}). 
\fcnrefsys{socket} returns an integer-valued descriptor or
``handle'' for the socket if successful.
If \fcnrefsys{socket} fails, it returns --1, and
we call our error handling function, \fcnref{DieWithSystemMessage}
(described later), to
print an informative hint and exit.

\tlcitems{Prepare address and establish connection}{27--40}

\begin{bottomcode}

\blcitems{Prepare \typesys{sockaddr\_in} structure to hold server address}{28--29}
To connect a socket, we have to specify the address and port to
connect to.  The \typesys{sockaddr\_in} structure is defined to be a
``container'' for this information.  The call to \fcnrefsys{memset()}
ensures that any parts of the structure that we do not explicitly set
contain zero.
% This statement is not useful.
%(This step is needed on some systems, but not all.)

\blcitems{Filling in the \typesys{sockaddr\_in}}{30--36}

We must set the address family (\constsys{AF\_INET}), Internet
address, and port number.  The function \fcnrefsys{inet\_pton()}
converts the string representation of the server's Internet address
(passed as a command-line argument in dotted-quad notation)
into a 32-bit binary representation.  The server's port
number was converted from a command-line string to binary earlier;
the call to \fcnrefsys{htons()} (``host to network short'')
ensures that the binary value is formatted as required by the API.
(Reasons for this are described in Chapter~\ref{chap:encoding}).

\blcitems{Connecting}{38--40}

The \fcnrefsys{connect()} function establishes a
connection between the given socket and the one identified by the address
and port in the \typesys{sockaddr\_in} structure.  Because
the sockets API is
generic, the pointer to the \typesys{sockaddr\_in} address
structure (which is specific to IPv4 addresses)
needs to be \emph{cast\/} to the generic type (\typesys{sockaddr}),
and the actual size of the address data structure must be supplied.

\end{bottomcode}

\tlcitems{Send echo string to server}{42--50}

We find the length of the argument string and save it for later use.
A pointer to the echo string is passed to the
\fcnrefsys{send()} call; the string itself was stored somewhere (like all
command-line arguments) when the application was started.  We do not
really care where it is, we just need to know the address of the first
byte and how many bytes to send.  (Note that we do
not send the end-of-string marker character (\termchar) that is at the end of
the argument string---and all strings in C).
\fcnrefsys{send()} returns the
number of bytes sent if successful and --1 otherwise.  If \fcnrefsys{send()}
fails or sends the wrong number of bytes, we must deal with the error.  Note
that sending the wrong number of bytes will not happen here.
Nevertheless, it's a good idea to include
the test, because errors can occur in some contexts.

\tlcitems{Receive echo server reply}{52--69}

TCP is a byte-stream protocol.  One implication of this type of
protocol is that \fcnrefsys{send()} boundaries are not preserved.
In other words:
\callout{The bytes sent by a call to \fcnrefsys{send()} on one end of
a connection may not all be returned by a single call to
\fcnrefsys{recv()} on the other end.}
(We discuss this issue in more detail in Chapter~\ref{chap:under}.)
So we need to repeatedly receive bytes until we have received as many
as we sent.  In all likelihood, this loop will only be executed once
because the data from the server will in fact be returned all at once;
however, that is not \emph{guaranteed} to happen, and so we have to
allow for the possibility that multiple reads are required.  This is a
basic principle of writing applications that use sockets:
\callout{you must never assume anything about what the network and the
program at the other end are going to do}.

\begin{bottomcode}

\blcitems{Receive a block of bytes}{56--63}

\fcnrefsys{recv()} blocks until data is available, returning the
number of bytes copied into the buffer or $-1$ in case of failure.  A
return value of zero indicates that the application at the other end
closed the TCP connection.  Note that the size parameter passed to
\fcnrefsys{recv()} reserves space for adding a terminating null character.

\blcitems{Print buffer}{65--66}

We print the data sent by the server as it is received.  We add the
terminating null character (\termchar) at the end of each chunk of
received data so that it can be treated as a string by
\fcnref{fputs()}.  We do not check whether the bytes received are the
same as the bytes sent. The server may send something completely
different (up to the length of the string we sent), and it will be
written to the standard output.

\blcitem{Print newline}{69}

When we have received as many bytes as we sent, we exit the loop and
print a newline.

\end{bottomcode}

\tlcitems{Terminate connection and exit}{71--72}

The \fcnrefsys{close()} function informs the remote socket that
communication is ended, and then deallocates local resources
of the socket.

\end{topcode}

Our client application (and indeed all the programs in this book)
makes use of two error handling functions:
\begin{quote}
\fcn{DieWithUserMessage(\type{const char *}\param{msg}, \type{const
char *}\param{detail})} \\
\fcn{DieWithSystemMessage(\type{const char *}\param{msg})} \\
\end{quote}
We call the first when an error is detected by our code, and the
second when an API function call returns a failure indication.
Both functions print a user-supplied message string (\param{msg}) to
\var{stderr}, followed by a detail message string; they then
call \fcnsys{exit()} with an error return code, causing the
application to terminate.
%
The only difference is the source of the detail message.  For
\fcn{DieWithUserMessage}, the detail message is user-supplied.  For
\fcn{DieWithSystemMessage},
the detail message is supplied by the system based on the
value of the special variable \varsys{errno}  (which describes the reason for
the most recent failure, if any, of a system call).  We call 
\fcn{DieWithSystemMessage} only if the error situation results from a
call to a system call that sets \varsys{errno}.
%
(To keep our programs simple, our examples do not contain much code
devoted to recovering from errors---they simply punt and exit.
Production code generally should not give up so easily.)

Occasionally we need to supply informatin to the user without exiting;
we use \fcnrefsys{printf()} if we need formatting capabilities, and
\fcnrefsys{fputs()} otherwise.  In particular, we try to avoid using
\fcn{printf()} to output fixed, preformatted strings.
\callout{One thing that you
should \emph{never\/} do is to pass text received from the
network as the first argument to \fcn{printf()}.  It creates a serious
security vulnerability.  Use \fcn{fputs()} instead.}

\begin{quote}
\textbf{Note:} the \fcn{DieWith\ldots()}
functions are \emph{declared\/} in the header
``\file{Practical.h}.'' However, the actual \emph{implementation\/}
of these functions is contained in the file \file{DieWithMessage.c},
which should be compiled and linked with \emph{all} example
applications in this text. 
\end{quote}

\jcode{DieWithMessage.c}{code/DieWithMessage.c}{1}{1}

If we compile \file{TCPEchoClient4.c} and \file{DieWithMessage.c} to
create program \exec{TCPEchoClient4}, we can communicate with
an echo server with Internet address \serverIPiv\ as follows:

\begin{shell}
\prompt \typed{TCPEchoClient4 \serverIPiv\ "Echo this!"} \\
\response{Received: Echo this!}
\end{shell}

For our client to work, we need a server.  Many
systems include an echo server for debugging and testing purposes; however, for
security reasons, such servers are often initially disabled.  If you don't have
access to an echo server, that's okay because we're about to write one.

\subsection{TCP Server}

\noindent We now turn our focus to constructing a TCP server.  The
server's job is to set up a communication endpoint and passively wait
for a connection from the client.
%As with clients, the setup for a
%TCP and UDP server is similar.
%For now we focus on a TCP server and
%discuss the differences with a UDP server in
%Chapter~\ref{chap:datagramsockets}.
There are four general steps for basic TCP server communication:

\begin{enumerate}

\item Create a TCP socket using \fcnrefsys{socket()}.

\item Assign a port number to the socket with \fcnrefsys{bind()}.

\item Tell the system to allow connections to be made to that
port, using \fcnrefsys{listen()}.

\item Repeatedly do the following:

\begin{itemize}

\item Call \fcnrefsys{accept()} to get a new socket for each client
connection.

\item Communicate with the client via that new socket
using \fcnrefsys{send()} and \fcnrefsys{recv()}.

\item Close the client connection using \fcnrefsys{close()}.

\end{itemize}

\end{enumerate}
Creating the socket, sending, receiving, and closing are the same as
in the client.  The differences  in the server's use of sockets
have to do with binding
an address to the socket and then using the socket as a channel to
obtain other sockets that are connected to clients.  (We'll elaborate
on this in the comments following the code.)
%
The server's communication with each client is as simple as can be: it simply
receives data on the client connection and sends the same data
back over to the client; it repeats this until the client closes its
end of the connection, at which point no more data will be forthcoming.

%% Creating the socket, sending, receiving and closing 
%% Our next example program, \file{TCPEchoServer4.c}, implements the
%% IPv4 TCP echo 
%% service used by our client program.  The server is simple.  Like most
%% servers, it runs forever, doing the following: (1)~it sets up a
%% listening socket and waits for an incoming connection from a client;
%% (2)~when one arrives, it repeatedly receives bytes and sends them
%% again until the client terminates the connection, and (3)~it closes
%% the client connection.

\jcode{TCPEchoServer4.c}{code/TCPEchoServer4.c}{1}{1}

\begin{topcode}

\tlcitems{Program setup and parameter parsing}{1--15}

We convert the port number from string to numeric value using
\fcn{atoi()}; if the first argument is not a number, \fcn{atoi()} will
return 0, which will cause an error later when we call \fcnsys{bind()}.

\tlcitems{Socket creation and setup}{17--35}

\begin{bottomcode}

\blcitems{Create a TCP socket}{17--20}

We create a stream socket just like we did in the client.

\blcitems{Fill in desired endpoint address}{22--27}

On the server, we need to associate our server socket with an address
and port number so client connections get to the right place.
Since we are writing for IPv4, we use a \typesys{sockaddr\_in}
structure for this.  Because we don't much care which address we are
on (any one assigned to the machine the
server is running on will be OK), we let the system pick it by
specifying the wildcard address \constsys{inaddr\_any} as our desired
Internet address.  (This is
usually the right thing to do for servers, and it saves the server from having
to find out any actual Internet address.)  Before setting both address
and port number in the \typesys{sockaddr\_in}, we
convert each to to network byte order using \fcnrefsys{htonl()} and
\fcnrefsys{htons()}.  (See Section~\ref{sect:byteordering} for details).

\blcitems{Bind socket to specified address and port}{29--31}

As noted above, the server's socket needs to be associated with a
local address and port; the function that accomplishes this
is \fcnrefsys{bind()}.  Notice that \callout{while the client has to
supply the server's address to \fcnrefsys{connect()}, the
server has to specify its \emph{own\/} address to
\fcnrefsys{bind()}}. It is this 
piece of information (i.e., the server's address and port) that they
have to agree on to communicate; neither one really needs to know the
client's address.
%
Note that \fcnrefsys{bind()} may fail for
various reasons; one of the most important is that some other socket
is already bound to the specified port (see Section~\ref{sect:demux}).
Also, on some systems special privileges are required to bind to certain
ports (typically those with numbers less than 1024).

\blcitems{Set the socket to listen}{33--35}

The \fcnrefsys{listen()} call tells the TCP implementation to allow incoming
connections from clients.  Before the call to \fcnrefsys{listen()},
any incoming connection requests to the socket's address would be silently
rejected---i.e., the \fcnrefsys{connect()} would fail at the client.

\end{bottomcode}

\tlcitems{Iteratively handle incoming connections}{37--56}

\begin{bottomcode}

\blcitems{Accept an incoming connection}{38--45}

As discussed above, a TCP socket on which \fcnrefsys{listen()} has
been called is used differently than the one we saw in the client
application.  Instead of sending and receiving on the socket, the
server application calls \fcnrefsys{accept()}, which blocks until an
incoming connection is made to the listening socket's port number.
At that point \fcnrefsys{accept()} returns a descriptor for a new
socket, which is already connected to the initiating remote socket.  The second
argument points to a \typesys{sockaddr\_in} structure, and the third
argument is a pointer to the length of that structure.  Upon success,
the \typesys{sockaddr\_in} contains the Internet address and port of
the client to which the returned socket is connected; the address's
length has been written into the integer pointed to by the third
argument.  Note that the socket referenced by the returned descriptor
is already \emph{connected}; among other things this means it is ready
for sending and receiving.  (For details about what happens in the
underlying implementation, see Section~\ref{sect:connecting} in
Chapter~\ref{chap:under}.)

\blcitems{Report connected client}{49--53}

At this point
\var{clntAddr} contains the address and port number
of the connecting client; we
provide a ``Caller ID'' function and print out the client's  information.
As you might expect, \fcnrefsys{inet\_ntop()} is the inverse of
\fcnrefsys{inet\_pton()}, which we
used in the client.  It takes the binary representation of the
client's address and converts it to a dotted-quad string.  Because the
implementation deals with ports and addresses in so-called network
byte order (Section~\ref{sect:byteordering}), we have to convert the
port number before passing it to \fcnsys{printf()} (\fcn{inet\_pton()}
takes care of this transparently for addresses).

\blcitem{Handle echo client}{55}

\fcnref{HandleTCPClient()} takes care of the ``application
protocol''.  We discuss it below.

 Thus, we have factored out the
``echo''-specific part of the server.

\end{bottomcode}

\end{topcode}

We have factored out the function that implements the ``echo'' part of
our echo server.
% 
Although this \emph{application protocol\/}
only takes a few lines to implement, it's good design
practice to isolate its details from the rest of the server code.
This promotes code re-use.

\fcnref{HandleTCPClient()} receives data on the
given socket and sends 
it back on the same socket, iterating as long as \fcnrefsys{recv()}
returns a positive value (indicating that something was received).
\fcnrefsys{recv()} blocks until something is
received or the client closes the connection.  When the client closes
the connection normally, \fcnrefsys{recv()} returns~\returncode{0}.  You can
find \fcnref{HandleTCPClient()} in the file \file{TCPServerUtility.c}.

\jcode[100]{HandleTCPClient()}{code/TCPServerUtility.c}{1}{73}

Suppose we compile \file{TCPEchoServer4.c}, \file{DieWithMessage.c}, and
\file{TCPServerUtility.c} into the executable program 
\exec{TCPEchoServer4}, and run that program on a host with Internet
(IPv4) address \serverIPiv, port 5000.  Suppose also that we
run our client on a host with Internet address \clientIPiv, and
connect it to the server.  The server's output should look 
like this:

\begin{shell}
\prompt \typed{TCPEchoServer4 5000} \\
\response{Handling client \clientIPiv}
\end{shell}

While the client's output looks like this:

\begin{shell}
\prompt \typed{TCPEchoClient4 \serverIPiv\ "Echo this!" 5000} \\
\response{Received: Echo this!}
\end{shell}

The server binds its socket to port 5000 and waits for a connection
request from the client.  The client connects, sends the message
``Echo this!'' to the server, and receives the echoed response.  In
this command we have to supply
\exec{TCPEchoClient} with the port number on the command line
because it is talking to our echo server, which is on port
5000 rather than the well-known port~7.

We have mentioned that a key principle for coding network applications
using sockets is \callout{Defensive Programming: your code must not make
assumptions about anything received over the network}.
%
What if you want to ``play'' with your TCP server to see how it responds to
various incorrect client behaviors?
You could write a TCP client that sends bogus messages and
prints results; this, however, can be tedious and time-consuming.  A quicker
alternative is to use the \typed{telnet} program available on most systems. 
This is a command-line tool that connects to a server, sends whatever text you
type, and prints the response.  Telnet takes two parameters:  the server and
port.  For example, to telnet to our example echo server from above, try

\begin{shell}
\prompt \typed{telnet \serverIPiv\ 5000}
\end{shell}

\noindent  Now type your string to echo and telnet will print the server
response.  The behaviour of telnet differs between implementations so
you may need to research the specifics of its use on your system.

Now that we've seen a complete client and server, let's look at
the individual functions that make up the Sockets API in a bit more detail.

\section{Creating and Destroying Sockets}

\noindent To communicate using TCP or UDP, a program begins by asking
the operating system to create an instance of the socket abstraction.
The function that accomplishes this is \fcnrefsys{socket()};
its parameters specify the flavor of socket needed by the program.

\begin{inlinefcn}
\type{int} \fcnsys{socket}(\type{int} \param{domain}, 
\type{int} \param{type}, \type{int} \param{protocol})
\end{inlinefcn}

\noindent The first parameter determines the communication \defn{domain} of
the socket.  Recall that the sockets API provides a generic interface
for a large number of communication domains; however, we are only interested
in IPv4 (\constsys{af\_inet}) and IPv6 (\constsys{af\_inet6}).  Note that you
may see some programs use \constsys{pf\_xxx} here instead of
\constsys{af\_xxx}.  Typically, these values are equal, in which case they are
interchangable, but this is (alas) not guaranteed.\footnote{Truth be
told, this is an ugly part of the Sockets interface, and the man pages
are simply not helpful.  It is possible that some 
of the code we will present later would fail if \constsys{pf\_*} and
\constsys{af\_*} were not the same.}
% We use \constsys{AF\_XXX} because
% 1) in our experience the AF variant is more common in code, 2) it is more
% consistent with man page and RFC documentation, and 3) the \typesys{addrinfo}
% structure contains a single family field, which according to RFC 3493 is
% AF_XXXX.

The second parameter specifies the \emph{type} of the socket.  The
type determines the semantics of data transmission with the
socket---for example, whether transmission is reliable, whether
message boundaries are preserved, and so on.  The constant
\constsys{sock\_stream} specifies a socket with reliable byte-stream
semantics, whereas \constsys{sock\_dgram} specifies a best-effort
datagram socket.

The third parameter specifies the particular \emph{end-to-end
protocol} to be used. For both IPv4 and IPv6, we
want TCP (identified by the constant \constsys{ipproto\_tcp}) for a
stream socket, or UDP (identified by \constsys{ipproto\_udp}) for a
datagram socket.  Supplying the constant 0 as the third parameter
causes the system to select the \emph{default} end-to-end protocol for
the specified protocol family and type.
Because there is currently only one choice
for stream sockets in the TCP/IP protocol family, we
could specify 0 instead of giving the protocol number explicitly.
Someday, however, there might be other end-to-end protocols in the
Internet protocol family that implement the same semantics.  In that
case, specifying 0 might result in the use of a different protocol,
which might or might not be desirable. The main thing is to ensure
that the communicating programs are using the same end-to-end
protocol.

We said earlier that \fcnrefsys{socket} returns a \emph{handle\/} for
the communication instance.  On Unix-derived systems, it is an integer:
a nonnegative value for success and $-1$ for failure.  A nonfailure
value should be treated as an opaque handle, like a file
descriptor. (In reality, it \emph{is} a file descriptor, taken from
the same space as the numbers returned by \fcnrefsys{open()}.)  This
handle, which we call a \defn{socket descriptor}, is passed to other
API functions to identify the socket abstraction on which the
operation is to be carried out.

When an application is finished with a socket, it calls
\fcnrefsys{close()}, giving the descriptor for the socket that is no
longer needed.

\begin{inlinefcn}
\type{int} \fcnsys{close}(\type{int} \param{socket})
\end{inlinefcn}

\noindent \fcnrefsys{close()} tells the underlying protocol stack to
initiate any actions required to shut down communications and
deallocate any resources associated with the socket.
\fcnrefsys{close()} returns \returncode{0} on success or
\returncode{-1} on failure.  Once \fcnrefsys{close()} has been called,
it is not possible to invoke any other operations (e.g.,
\fcnrefsys{send()} and \fcnrefsys{recv()} on the socket.

\section{Specifying Addresses}
\label{sect:specifyingAddresses}

\noindent Applications using sockets need to be able to identify the
remote endpoint(s) with which they will communicate.
We've already seen that a client must
specify the address and port number
of the server application with which it needs to
communicate.  In addition, the sockets layer sometimes needs to pass
addresses to the application.  For example, a feature analogous to
``Caller ID'' in the telephone network lets a server know the
address and port number of each client that communicates with~it.

In this section we describe the data structures used as containers for
this information by the sockets API.

\subsection{Generic Addresses}

The sockets API defines a generic data type---the \typesys{sockaddr}
structure---for specifying addresses associated with sockets:

\begin{inlinecode}
struct sockaddr {
    sa_family_t sa_family;     // Address family (e.g., AF_INET)
    char sa_data[14];          // Family-specific address information
};
\end{inlinecode}

\noindent The first part of this address structure defines the address
family---the space to which the address belongs.  For our purposes, we
will always use the system-defined
constants \constsys{af\_inet} and \constsys{af\_inet6},
which specify the Internet address families for IPv4 and IPv6,
respectively.
%
The second part is a blob of bits whose exact form
depends on the address family.  (This is a typical way of
dealing with heterogeneity in operating systems and networking.)  As we
discussed in Section~\ref{sect:introaddr}, socket addresses for the
Internet protocol family have two
parts: a 32-bit (IPv4) or 128-bit (IPv6) Internet address and a 16-bit port
number.\footnote{The astute reader may have noticed that the generic
\type{sockaddr} structure is not big enough to hold both a 16-byte IPv6
address and a 2-byte port number.  We'll deal with this 
difficulty shortly.}

\subsection{IPv4 Addresses}

The particular form of the \typesys{sockaddr} structure that is used
for TCP/IP socket addresses depends on the IP version.  For IPv4, use the
\typesys{sockaddr\_in} structure.

\begin{inlinecode}
struct in_addr {
    uint32_t s_addr;       // Internet address (32 bits)
};

struct sockaddr_in {
    sa_family_t sin_family;    // Internet protocol (AF_INET)
    in_port_t sin_port;        // Address port (16 bits)
    struct in_addr sin_addr;   // IPv4 address (32 bits)
    char sin_zero[8];          // Not used
};
\end{inlinecode}

As you can see, the \typesys{sockaddr\_in} structure has fields for
the port number and Internet address in addition to the address
family.  It is important to understand that \typesys{sockaddr\_in} is
just a more detailed view of the data in a \typesys{sockaddr} structure,
tailored to sockets using IPv4.  Thus, we can fill in the
fields of a \typesys{sockaddr\_in} and then cast (a pointer to) it to
a (pointer to a) \typesys{sockaddr} and
pass it to the socket functions, which look at the
\structmem{sa\_family} field to learn the actual type, then cast back
to the appropriate type.

\subsection{IPv6 Addresses}

For IPv6, use the \typesys{sockaddr\_in6} structure.

\begin{inlinecode}
struct in_addr {
    uint32_t s_addr[16];       // Internet address (128 bits)
};

struct sockaddr_in6 {
    sa_family_t sin6_family;   // Internet protocol (AF_INET6)
    in_port_t sin6_port;       // Address port (16 bits)
    uint32_t sin6_flowinfo;    // Flow information
    struct in6_addr sin6_addr; // IPv6 address (128 bits)
    uint32_t sin6_scope_id;    // Scope identifier
};
\end{inlinecode}

The \typesys{sockaddr\_in6} structure has
additional fields beyond those of a \typesys{sockaddr\_in}.
These are intended for capabilities of the IPv6 protocol that are not
commonly used.  They will be (mostly) ignored in this book.

As with \typesys{sockaddr\_in},
we must cast (a pointer to) the \typesys{sockaddr\_in6 *} to (a
pointer to) a \typesys{sockaddr *} in
order to pass it to the various socket functions;  again,
the address family field is used by the implementation to
determine the actual type of the argument.

\subsection{Generic Address Storage}

If you know anything about how data structures are allocated in C,
you may have already noticed that a \typesys{sockaddr} is really not big enough
to hold a \typesys{sockaddr\_in6}.  (If you don't know anything about
it, don't fear: much of what you need to know will be covered in
Chapter \ref{chap:encoding}.)  In particular,
what if we want to allocate an address structure, but we don't know
the actual address type (e.g., IPv4 or IPv6)?  The 
generic \typesys{sockaddr} won't work because it's too small for some
address structures.\footnote{You may wonder why this is so (we do).  The
  reasons apparently have to do with backward-compatibility: the
  sockets API was first specified a long time ago, before IPv6, when
  resources were scarcer and
  there was no reason to have a bigger structure.  Changing it now to
  make it bigger   would apparently break binary-compatibility with
  some applications.} 
To solve this problem, the socket designers created the
\typesys{sockaddr\_storage} structure, which is 
guaranteed to be as large as any supported address type.

\begin{inlinecode}
struct sockaddr_storage {
    sa_family_t ss_family;     // Address family (e.g., AF_INET)
    ...
    // Padding and fields to get correct length and alignment
    ...
};
\end{inlinecode}

\noindent As with \typesys{sockaddr}, we still have the leading family field to
determine the actual type of the address; however, with
\typesys{sockaddr\_storage} we have sufficient space for any address
type.
(For a hint about how this could be accomplished, refer to the
discussion of how the C compiler lays out structures in memory, in
\Sect{sect:overlay}.)

One final note on addresses.  On some platforms, the address structures
contain an additional field that stores the length of the address structure in
bytes.  For \typesys{sockaddr}, \typesys{sockaddr\_in}, \typesys{sockaddr\_in6},
and \typesys{sockaddr\_storage}, the extra fields are called
\structmem{sa\_len}, \structmem{sin\_len}, \structmem{sin6\_len}, and
\structmem{ss\_len}, respectively.  Since a length field is not available on all
systems, avoid using it.  Typically, that use this form of structure
define a value (e.g., \const{sin6\_len}) that can be tested
for at compile time to see if the length field is present.

%\begin{figure}[htb]
%\begin{center}
%\jfig{figures/sockaddr.eps}{,width=5.5in}
%\caption{Address structures\label{fig:addrstructs}}
%\end{center}
%\end{figure}

%\float{figure}
%\art{figure}{do02f01,28.42pc,8pc,100}
%\caption{Address structures.}
%\tag{fig:addrstructs}
%\endfloat

\subsection{Binary/String Address Conversion}

For socket functions to understand addresses, they must be in
``numeric'' (i.e., binary) form; however, addresses for human use are generally
``printable'' strings (e.g., 192.168.1.1 or 1::1).  We can convert addresses
from printable string to numeric using the \fcnrefsys{inet\_pton()}
function (\textbf{pton} = \textbf{p}rintable \textbf{to} \textbf{n}umeric):

\begin{inlinefcn}
\type{int} \fcnrefsys{inet\_pton}(\type{int} \param{addressFamily}, 
\typesys{const char *}\param{src}, \type{void *}\param{dst})
\end{inlinefcn}

\noindent The first parameter, \defn{addressFamily}, specifies the
address family of the address being converted.
Recall that the sockets API provides a generic
interface for a large number of communication domains; however,
we are only interested here
in IPv4 (\constsys{af\_inet}) and IPv6 (\constsys{af\_inet6}). 
The \param{src} parameter references a null-terminated character string
containing the address to convert.  The \param{dst} parameter points
to a block of memory in the caller's space to hold the result; its
length must be sufficient to hold the result (at least 4 bytes for IPv4 and 16
bytes for IPv6).
%
\fcnrefsys{inet\_pton()} returns \returncode{1} if the conversion
succeeds with the address referenced by \param{dst} in network byte
order; \returncode{0} if the string pointed to by \param{src} is not
formatted as a valid address;
and \returncode{-1} if the specified address family is unknown.

We can go the other way, converting addresses from numeric to
printable form, using \fcnrefsys{inet\_ntop()}
(\textbf{ntop} = \textbf{n}umeric \textbf{to} \textbf{p}rintable):

\begin{inlinefcn}
\type{const char *}\fcnrefsys{inet\_ntop}(\type{int} \param{addressFamily}, 
\typesys{const void *}\param{src}, \type{char *}\param{dst}, \type{socklen\_t}
\param{dstBytes})
\end{inlinefcn}

\noindent The first parameter, \defn{addressFamily}, specifies the
type of the address being converted. The second parameter
\param{src} points to the first byte  of a block of memory containing
the numeric address to convert.  The size of the block
is determined by the address family.  The \param{dst} parameter points
to a buffer (block of memory) allocated in the caller's space, into which the
resulting string will be copied; its size is given by
\param{dstBytes}.
How do we know what size to make the block of memory?
The system-defined constants
\constsys{inet\_addrstrlen} (for IPv4) and
\constsys{inet6\_addrstrlen} (for IPv6) indicate the longest possible
resulting string (in bytes).  \fcnrefsys{inet\_ntop()}
returns a pointer to the string containing the printable address
(i.e., the third argument) if the conversion succeeds and NULL otherwise.

\subsection{Getting a Socket's Associated Addresses}
\label{sect:getpeername}
The system associates a local and foreign address with each
connected socket (TCP or UDP).
Later we'll discuss the details of how these values are assigned.
We can find out these addresses using \fcnrefsys{getsockname()}
for the local address and \fcnrefsys{getpeername()} for the foreign
address.  Both
methods return a \type{sockaddr} structure containing the Internet address
and port information.

\begin{inlinefcn}
\type{int} \fcnrefsys{getpeername}(\type{int} \param{socket},
\typesys{struct sockaddr *}\param{remoteAddress}, 
\typesys{socklen\_t *}\param{addressLength})\\
\type{int} \fcnrefsys{getsockname}(\type{int} \param{socket},
\typesys{struct sockaddr *}\param{localAddress}, 
\typesys{socklen\_t *}\param{addressLength})
\end{inlinefcn}

The \param{socket} parameter is the descriptor of the socket whose
address information we want. The  \param{remoteAddress} and
\param{localAddress} parameters point to address structures
into which the address information will be placed by the
implementation; they are always cast to
\type{sockaddr *} by the caller.
\callout{If we don't know the IP protocol version a priori, we
should pass in a (pointer to a)
\typesys{sockaddr\_storage} to receive the result.}  As
with other socket calls using \type{sockaddr}, the \param{addressLength} is an
in-out parameter specifying the length of the buffer (input) and
returned address structure (output) in bytes.

\section{Connecting a Socket}
\label{sect:connectingASocket}

A TCP socket must be connected to another socket before any data can
be sent through it.  In this sense using TCP sockets is something like
using the telephone network. Before you can talk, you have to specify
the number you want, and a connection must be established; if the
connection cannot be established, you have to try again later.  The
connection establishment process is the biggest difference between
clients and servers: The client initiates the connection while the
server waits passively for clients to connect to it.  (For additional
details about the connection process and how it relates to the API
functions, see Section~\ref{sect:lifecycle}.)  To establish a
connection with a server, we call \fcnrefsys{connect()} on the
socket.

\begin{inlinefcn}
\type{int} \fcnrefsys{connect}(\type{int} \param{socket}, 
\typesys{struct sockaddr *}\param{foreignAddress}, 
\type{unsigned int} \param{addressLength})
\end{inlinefcn}

The first argument, \param{socket}, is the descriptor created by
\fcnrefsys{socket()}.  \param{foreignAddress} is declared to
be a pointer to a \typesys{sockaddr} because the sockets
API is generic; for our purposes, it will always be a pointer to either a
\typesys{sockaddr\_in} or \typesys{sockaddr\_in6} containing the Internet
address and port of the server.  \param{addressLength} specifies the
length of the address structure, typically given as
\fcnrefsys{sizeof(struct sockaddr\_in)} or \fcnrefsys{sizeof(struct
sockaddr\_in6)}. When \fcnrefsys{connect()} returns, the socket is connected and
communication can proceed with calls to \fcnrefsys{send()} and
\fcnrefsys{recv()}.

\section{Binding to an Address}
\label{sect:bindingToAnAddress}

As we have noted already, client and server ``rendezvous'' at the
server's address and port.  For that to work, the server must first be
associated with that address and port.
This is accomplished using \fcnrefsys{bind()}.
Again, note that the client supplies
the server's address to \fcnrefsys{connect()}, but the
server has to specify its \emph{own\/} address
to \fcnrefsys{bind()}.
Neither client nor server application needs to know the 
client's address in order for them to communicate.  (Of course, the
server may wish to know the client's address for logging or other
purposes.) 

\begin{inlinefcn}
\type{int} \fcnrefsys{bind}(\type{int} \param{socket}, 
\typesys{struct sockaddr *}\param{localAddress}, 
\type{unsigned int} \param{addressSize})
\end{inlinefcn}

The first parameter is the descriptor returned by an earlier call to
\fcnrefsys{socket()}.  As with \fcnrefsys{connect()}, the address
parameter is declared as a pointer to a \typesys{sockaddr}, but for
TCP/IP applications, it will always point to a
\typesys{sockaddr\_in} (for IPv4) or \typesys{sockaddr\_in6} (for IPv6),
containing the Internet address of the local interface and the port to
listen on.
The \param{addressSize} parameter is the size
of the address structure.  \fcnrefsys{bind()} returns
\returncode{0} on success and \returncode{-1} on failure.

It is important to realize that it is not possible for a program to
bind a socket to an \emph{arbitrary\/} Internet address---if a
specific Internet address is given (of either type),
the call will only succeed if that
address is assigned to the host on which the program is running.
A server on a host with
multiple Internet addresses might bind to a specific one because it
\emph{only wants to accept connections that arrive to that address}.
Typically, however, the server wants to accept connections sent to
\emph{any\/} of the host's addresses, and so sets
the address part of the \type{sockaddr} to
the ``wildcard'' address \constsys{inaddr\_any} for IPv4 or 
\constsys{in6addr\_any} for IPv6.  The semantics of the wildcard
address are that it matches any specific address.  For a server, this
means that it will receive connections addressed to any of the host's
addresses (of the specified type).

While
\typesys{bind()} is mostly used by servers, a client can also use
\typesys{bind()} to specify its local address/port.  For those
TCP clients that don't pick their own local address/port with
\typesys{bind()}, the local 
Internet address and port are determined
during the call to \typesys{connect()}.   Thus, a
client must call \typesys{bind()} \emph{before\/} calling
\typesys{connect()} if it is going to use it.

You can initialize a \typesys{in6\_addr} structure to the
wildcard address with \constsys{in6addr\_any\_init}; however, this
special constant may only
be used as an ``initializer'' in a declaration.
\callout{Note well that while \constsys{inaddr\_any} is
defined to be
in host byte order and, consequently, must be converted to network byte
order with \fcnrefsys{htonl()} before being used as an argument to
\fcnsys{bind()}, \constsys{in6addr\_any}
and \constsys{in6addr\_any\_init} are already in network byte order.}

Finally, if you supply the port number 0 to \typesys{bind()}, the
system will select an unused local port for you.

%% For connected sockets where the system is allowed to select, the local
%% Internet address and/or port are assigned on \typesys{connect()} or to
%% new sockets returned by \typesys{accept()}.  For example, in
%% \file{TCPEchoClient4.c} the system selects both the local Internet
%% address and port on \typesys{connect()}.  In \file{TCPEchoServer4.c}
%% the system selects only the local Internet address.

\section{Handling Incoming Connections}
\label{sect:handlingIncomingConnections}

After binding, the server socket has an address (or at least a
port).  Another step is required to
instruct the underlying protocol implementation to listen for 
connections from clients; this is done by calling \fcnrefsys{listen()}
on the socket.

\begin{inlinefcn}
\type{int} \fcnrefsys{listen}(\type{int} \param{socket}, 
\type{int} \param{queueLimit})
\end{inlinefcn}

The \fcnrefsys{listen()} function causes internal state changes to the
given socket, so that incoming TCP connection requests will be processed
and then queued for acceptance by the program
(Section~\ref{sect:lifecycle} in Chapter~\ref{chap:under} has more
details about the life cycle of a TCP connection).  The \param{queueLimit}
parameter specifies an upper bound on the number of incoming
connections that can be waiting at any time. The precise
effect of \param{queueLimit} is very system dependent, so consult your
local system's technical specifications.\footnote{See information in
  the preface about using ``man'' pages.}  \fcnrefsys{listen()} returns
\returncode{0} on success and \returncode{-1} on failure.

Once a socket is configured to listen, the program can begin accepting client
connections on it.  At first it might seem that a server should now wait for a
connection on the socket that it has set up, send and receive through that
socket, close it, and then repeat the process.  However, that is not
the way it works.  The socket that has been bound to a port and marked
``listening'' is never actually used for sending and receiving.
Instead, it is used as a way of getting \emph{new} sockets, one for
each client connection; the server then sends and receives on the
\emph{new} sockets.  The server gets a socket for an incoming client
connection by calling \fcnrefsys{accept()}.

\begin{inlinefcn}
\type{int} \fcnrefsys{accept}(\type{int} \param{socket}, 
\typesys{struct sockaddr *}\param{clientAddress}, 
\type{unsigned int *}\param{addressLength})
\end{inlinefcn}

This function dequeues the next connection on the queue for
\param{socket}.  If the queue is empty, \fcnrefsys{accept()} blocks
until a connection request arrives.  When successful,
\fcnrefsys{accept()} fills in the \typesys{sockaddr} structure
pointed to by \param{clientAddress} with the address and port
of the client at the other end of the connection.
Upon invocation, the \param{addressLength} parameter specifies the 
size of the structure pointed to by \param{clientAddress} (i.e., the
space available); upon return it
contains the size of the actual address returned.
\callout{A common beginner mistake (\#27) is to fail to initialize
  the integer that \param{addressLength} points to so it contains
 the length of the structure that \param{clientAddress} points to.}
Like this:

\begin{inlinecode}
  struct sockaddr_storage address;
  unsigned int addrLength = sizeof(address);
  int newConnection = accept(sock, &address, &addrLength);
\end{inlinecode}

If successful, \fcnrefsys{accept()} returns a descriptor for a
\emph{new} socket that is connected to the client.  The socket passed as
the first parameter to \fcnrefsys{accept()} is unchanged (not
connected to the client) and continues to listen for new connection
requests.  On failure, \fcnrefsys{accept()} returns \returncode{-1}.
On most systems, \fcnsys{accept()} only fails when passed a bad socket
descriptior.  However, on some platforms it may return an
error if the new socket has experienced a network-level error after
being created and before being accepted.

\section{Communication}

Once a socket is ``connected,'' you can begin sending and receiving data. 
As we've seen, a client creates a connected socket by calling
\fcnrefsys{connect()}, and a connected socket is returned by
\fcnrefsys{accept()} on a server.  After connection, the distinction
between client and server effectively disappears, at least as far as
the Sockets API is concerned.  Through a connected
TCP socket, you can communicate using \fcnrefsys{send()} and
\fcnrefsys{recv()}.

\begin{inlinefcn}
\type{int} \fcnrefsys{send}(\type{int} \param{socket}, 
\type{const void *}\param{msg}, \type{unsigned int} \param{msgLength}, 
\type{int} \param{flags})\\
\noindent\type{int} \fcnrefsys{recv}(\type{int} \param{socket}, 
\type{void *}\param{rcvBuffer}, \type{unsigned int} \param{bufferLength}, 
\type{int} \param{flags})
\end{inlinefcn}

These functions have very similar arguments.  The first parameter
\param{socket} is the descriptor for the connected socket through
which data is to be sent or received.  For \fcnrefsys{send()},
\param{msg} points to the sequence of bytes to be sent,
and \param{msgLength} is the number of bytes to send. The default behavior for
\fcnrefsys{send()} is to block until all of the data is sent. (We
revisit this behavior in \Sect{sect:nbio} and
Chapter~\ref{chap:under})
%
For \fcnrefsys{recv()}, \param{rcvBuffer} points to the buffer---that
is, an area in memory such as a character array---where received data
will be placed, and \param{bufferLength} gives the length of the
buffer, which is the maximum number of bytes that can be received at
once.  The default behavior for \fcnrefsys{recv()} is to block until
at least some bytes can be transferred.  (On most systems the minimum
amount of data that will cause the caller of \fcnsys{recv()} to
unblock is 1 byte.)

The \param{flags} parameter in both \fcnrefsys{send()} and
\fcnrefsys{recv()} provides a way to change some aspects of
the default behavior of
the socket call.  Setting \param{flags} to 0 specifies the default
behavior.  \fcnrefsys{send()} and \fcnrefsys{recv()} return the number
of bytes sent or received or \returncode{-1} for failure.  (See also
\Sect{sect:nbio}.)


Remember: TCP is a byte-stream protocol so \fcnrefsys{send()}
boundaries are not preserved.
\callout{The number of bytes read in a single call to
\fcnrefsys{recv} on the receiver is not necessarily determined by the number
of bytes written by a single call to \fcnrefsys{send()}.}  If you call
\fcnrefsys{send()} with 3000 bytes, it may take several calls to
\fcnrefsys{recv()} to get all 3000 bytes, even if you pass a 5000-byte
buffer to each \fcnsys{recv()} call. If you call \fcnrefsys{send()} with 100
byte four times, you might receive all 400 bytes with a single call to
\fcnrefsys{recv()}.  A common mistake when writing TCP socket applications
involves assuming that if you write all of the data with one \fcnrefsys{send()}
you can read it all with one \fcnrefsys{recv()}.  All these possibilities
are illustrated in Chapter~\ref{chap:under}.

\section{Using IPv6}

So far, we've seen a client and server that work only with IPv4.
What if you want to use IPv6? 
The changes are relatively minor and basically involve using the IPv6
equivalents for the address structure and constants.  Let's look at the IPv6
version of our TCP echo server.

\jcode{TCPEchoServer6.c}{code/TCPEchoServer6.c}{1}{1}

\begin{topcode}

\tlcitems{Socket creation}{17--20}

We construct an IPv6 socket by specifying the communication domain as
\constsys{af\_inet6}.

\tlcitems{Fill in local address}{22--27}

For the local address, we use the IPv6 (\typesys{struct
sockaddr\_in6}) address structure and constants (\constsys{af\_inet6} and
\constsys{in6addr\_any}).  One subtle difference is that we do
not have to convert \constsys{in6addr\_any} to network byte order as we did
with \constsys{inaddr\_any}.

\tlcitems{Report connected client}{38, 49--53}

\var{clntAddr}, which contains the address of the connecting client,
is declared as an IPv6 socket address structure.
When we convert the numeric address representation
to a string, the maximum string length is now \constsys{inet6\_addrstrlen}.
Finally, our call to \fcnrefsys{inet\_ntop()} uses an IPv6 address.

\end{topcode}

You've now seen both IPv4- and IPv6-specific clients and servers.
In Chapter~\ref{chap:addrindep} we will see how they can be made to work with
either type of address.

\begin{exercises}

\item Experiment with the book's TCP echo server using telnet.  What OS are you
using?  Does the server appear to echo as you type (character-by-character) or
only after you complete a line?

\item Use telnet to connect to your favorite web server on port 80 and fetching
the default page.  You can usually do this by sending the string ``GET /'' to
the web server.  Report the server address/name and the text from the default
page.

\item For \file{TCPEchoServer.c} we explicitly provide an address to
the socket using \fcnrefsys{bind()}.  We said that a socket
must have an address for communication, yet we do not perform a
\fcnrefsys{bind()} in \file{TCPEchoClient.c}.  How is the echo
client's socket given a local address?

\item Modify the client and server so that the server ``talks'' first,
  sending a greeting message,
  and the client waits until it has received the greeting before
  sending anything.  What needs to be agreed upon between client and
  server?

\item Servers are supposed to run for a long time without stopping.
Therefore, they have to be designed to provide good service no matter
what their clients do.  Examine the example \file{TCPEchoServer.c} and
list anything you can think of that a client might do to cause the
server to give poor service to other clients.  Suggest improvements to
fix the problems you find.

\item Using \fcnrefsys{getsockname()} and \fcnrefsys{getpeername()}, modify
\file{TCPEchoClient4.c} to print the local and foreign address immediately
after \fcnrefsys{connect()}.
\item What happens when you call \fcnrefsys{getpeername()} on an unconnected
TCP socket?

\item Using \fcnrefsys{getsockname()} and \fcnrefsys{getpeername()}, modify
\file{TCPEchoServer4.c} to print the local and foreign address for
the server socket immediately before and after \fcnrefsys{bind()} and for the
client socket immediately after it's returned by \fcnrefsys{accept()}.

\item Modify \file{TCPEchoClient4.c} to use \typesys{bind()} so that the system
selects both the address and port.

\item Modify \file{TCPEchoClient4.c} so that the new version binds to a
local Internet address with system-selected port.  What system changes would
break this code? Could you use this client on a different system?

\item What happens when you attempt to bind after calling \typesys{connect()}?

\item
Why does the socket interface use a special socket to accept
connections?  In other words, what would be wrong with having a server
create a socket, set it up using \fcn{bind()} and
\fcn{listen()}, wait for a connection, send and receive through
\emph{that\/} socket, and then when it is finished, close it and
repeat the process?  (Hint: think about what happens to connection
requests that arrive right after the server closes the previous connection.)
\end{exercises}

